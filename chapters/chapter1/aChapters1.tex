\chapter{Introduction}
\fxnote{Previous version: ``We, the group 630 on the Bachelors Degree in Control, wanted to work with an unstable system for this bachelor project in control engineering. The choice fell on a form of inverted pendulum setup that is called Cubli.''} 
Effective control of inverted pendulum is still an active area of research nowadays \cite{JHuber}. Choosing such a system as a Bachelor project in Control Engineering is therefore a stimulating but acceptable challenge\fxnote{find synonym for stimulating, and reword the challenge thing, B}.\\
The final choice fell on a form of inverted pendulum setup that is called Cubli. This structure is a cube that can jump up and balance on one of its edges or on one of its corners, as shown in \figref{CubliCorner}.
The Cubli is designed as a simple setup to let control engineers work with an inverted pendulum. A working Cubli can also be a fun thing to show the general public and explain what control engineering is about.  \cite{MGajamohan}

\begin{figure}[H] 
	\centering
	\includegraphics[scale=1.3]{figures/CubliCorner-700x430}
	\caption{A Cubli balancing on one of its corners\cite{RAndrea}}
	\label{CubliCorner}
\end{figure} 
The Cubli balances with the help of reaction wheels that spin up or down. Jumping up from a resting position is done by braking one or more of the reaction wheels, which overcomes the gravity and inertia keeping the Cubli down.
By controlling the direction of the jump-up motion and the falling direction, the Cubli can move around its body. At this point, the Cubli can be considered a cube robot.\\
Applications for this cube robot, that moves without any external tools, might seem limited when you only have a single cube. If you take a group of cubes, they could move together to traverse obstacles or solve puzzles one cube alone could not. A group of cubes can form a structure (\figref{MBlocksExample}), and by talking in between each other they can use their reactions wheels to get the structure they are forming to move in the desired direction. Since each cube can move independently a single cube can detach for an assignment or catch up with the main structure if it gets dropped.\cite{JRomanishin}

\begin{figure}[H] 
	\centering
	\includegraphics[scale=0.4]{figures/m-blocks}
	\caption{A number of cube robots (called M-blocks at Massachusetts Institute of Technology (MIT)) shown making two different structures. These M-blocks stick together with the help of magnets placed in their corners\cite{LRosen}}
	\label{MBlocksExample}
\end{figure} 




