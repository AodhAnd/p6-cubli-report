\section{Verification of the Model}

To verify the model in simulation \eqref{FrameEq4TaylerApprox} is transformed into the Laplace domain, after which a transfer function of the system is derived. The proceeding equations are valid only around the operating point, and so for better overview, in the following \si{\Delta \theta_F = \theta_F}.
%
\begin{flalign}
	\eq{(J_F+m_w \cdot {l_w}^{2}) \cdot \theta_F \cdot s^2}{-B_F \theta_F\cdot s +  ( m_F \cdot l_F + m_w \cdot l_w ) g \cdot \theta_F - \tau_m + B_w \theta_w\cdot s } & \nonumber\\
\label{LaplaceOfLinearizedModel}
\end{flalign}
%
The angle of the reaction wheel, \si{\theta_w}, still features in \eqref{LaplaceOfLinearizedModel}. It is desirable to have only one input, \si{\tau_m}, and one output, \si{\theta_F}. To achieve this, \eqref{WheelRotEq2} is transformed into the Laplace domain and solved for \si{\theta_w}.
%
\begin{flalign}
	\eq{\theta_w\cdot s^2} {\frac{\tau_m - B_w \theta_w\cdot s}{J_w} - \theta_F\cdot s^2}   &\\
	\eq{\theta_w} {\frac{ -J_w \theta_F \cdot s^2 + \tau_m }{ J_w \cdot s^2 + B_w \cdot s }}&
\label{WheelRotEq2Laplace}
\end{flalign}
%
\Eqref{WheelRotEq2Laplace} is now substituted for \si{\theta_w} in \eqref{LaplaceOfLinearizedModel}, and the transfer function is of the system is derived.
%
\begin{flalign}
	\eqOne{(J_F+m_w \cdot {l_w}^{2}) \cdot \theta_F \cdot s^2}{-B_F \theta_F\cdot s +  ( m_F \cdot l_F + m_w \cdot l_w ) g \cdot \theta_F - \tau_m}
	\eqTwo{+ B_w ( \frac{ -J_w \theta_F \cdot s^2 + \tau_m }{ J_w \cdot s^2 + B_w \cdot s } )\cdot s }&\nonumber
\label{CubliTransferFunction}
\end{flalign}

\vspace{-.2cm}
\large{\si{\frac{\theta_F}{\tau_m} =}}\nolinebreak
\Large{
\si{\frac{\frac{s}{-J_F - m_w \cdot {l_w}^2}}{s^3 + \left( \frac{B_w}{J_w} + \frac{B_w + B_F}{J_F + m_w \cdot {l_w}^2} \right) s^2 - \left( \frac{ \left( m_F \cdot l_F + m_w \cdot l_w \right)\cdot g}{ \left( J_F + m_w \cdot {l_w}^2 \right) J_w} - \frac{B_F B_w}{ \left(J_F + m_w \cdot {l_w}^2 \right) J_w} \right) s - \frac{\left(m_F \cdot l_F + m_w \cdot l_w \right) B_w\cdot g}{\left(J_F + m_w \cdot {l_w}^2 \right) J_w} }}}\normalsize\vspace{-1.9cm}\\
\vspace{1.8cm}\begin{flalign}\label{2ndCubliTransferFunction}\end{flalign}
%
The transfer function from \eqref{2ndCubliTransferFunction} can be represented as well in the form of a block diagram, as seen in \figref{cubliSimulink}.

% \begin{figure}[H] 
% 	\centering 
% 	\includegraphics[scale=0.53]{figures/cubliSimulink}
% \end{figure}
% \end{figure} 
\begin{figure}[H]
	  \begin{tikzpicture}[ auto,
                       thick,                         %<--setting line style
                       node distance=1.5cm,             %<--setting default node distance
                       scale=0.75,                     %<--|these two scale the whole thing
                       every node/.style={scale=0.62}, %<  |(always change both)
                       >=triangle 45 ]

    %-- Blocks creation --%
    \draw
      % DIRECT TERM %
      node[shape=coordinate][](input1) at (0,0){}
      node[shape=coordinate][](feedForward) at (0.5,0){}
      node(sum1) at (7.75,0) [sum] {$\sum$}
      node(sum2) at (9.25,0) [sum]{$\sum$}
      node(sum3) at (10.75,0) [sum]{$\sum$}

      node(torque2rotacc1) at (12.85,0) [block]{\Large $\frac{1}{J_F + m_w \cdot {l_w}^{2}}$}

      node(integration1) at (15.75,0) [block] {\Large $\frac{1}{s}$}
      node(integration2) at (18.2,0) [block] {\Large $\frac{1}{s}$}

      node[shape=coordinate][](output) at (19,0){}
      node[shape=coordinate][](veloFeedbackNode) at (16.8,0){}
      node[shape=coordinate][](accFeedbackNode) at (14.5,0){}
    ;
    \draw
      % REACTION WHEEL EQUATIONS %  
      node(sum4) at (1.5,-2) [sum]{$\sum$}
      node(sum5) at (2.85,-2) [sum]{$\sum$}

      node(torque2rotacc2) at (4.3,-2) [block]{\Large $\frac{1}{J_w \cdot s}$}
      % node(integration3) [block, right of = torque2rotacc2] {$\frac{1}{s}$}
      node(frictionWheel) at (6.9,-2) [block] {\Large $B_w$}

      node[shape=coordinate][](veloWheelFeedback) at (7.75,-3.5){}
    ;
    \draw
      % FEEDBACKS %
      node(accFeedback) at (4, -6) [block] {\Large $J_w$}
      node(veloFeedback) at (12.65,-2) [block] {\Large $B_F$}
      node(angleFeedback) at (11.65,-4) [block] {\Large $(m_F \cdot l_F + m_w \cdot l_w)g$}
    ;
    %-- Block linking --%
    % INPUT %
    \draw[-](input1)        -- node{\Large $\tau_m(s)$}(feedForward);
    \draw[->](feedForward)  -- (sum1);

    % OUTPUT %
    \draw[-](integration2)  -- (output);
    \draw[->](output)       -- node {\Large $\theta_{F}(s)$} (20,0);

    % DIRECT TERM %
    \draw[->] (sum1)            -- (sum2);
    \draw[->] (sum2)            -- (sum3);
    \draw[->] (sum3)            -- (torque2rotacc1);
    \draw[->] (torque2rotacc1)  -- node{\Large $\alpha_F(s)$}(integration1);
    \draw[->] (integration1)    -- node{\Large $\omega_F(s)$}(integration2);

    % REACTION WHEEL EQUATIONS %
    \draw[->] (feedForward)     |- (sum4);
    \draw[->] (sum4)            -- (sum5);
    \draw[->] (sum5)            -- (torque2rotacc2);
    \draw[->] (torque2rotacc2)  -- node{\Large $\omega_w(s)$}(frictionWheel);
    % \draw[->] (integration3)    -- (frictionWheel);
    \draw[->] (frictionWheel)   -| (sum1);

    \draw[-] (frictionWheel)       -| (veloWheelFeedback);
    \draw[->] (veloWheelFeedback)  -| (sum5);

    % FEEDBACKS
    \draw[->] (accFeedbackNode)  |- (accFeedback);
    \draw[->] (accFeedback)      -| (sum4);

    \draw[->] (output)           |- (angleFeedback);
    \draw[->] (angleFeedback)    -| (sum2);

    \draw[->] (veloFeedbackNode) |- (veloFeedback);
    \draw[->] (veloFeedback)     -| (sum3);

    %-- Nodes --%
    \draw%--------------------------------------------------------------
      node at (input1)            [shift={(-0.04, -0.05 )}] {\Large \textopenbullet}
      node at (output)            [shift={( 0.007, -0.05 )}] {\Large \textbullet}
      node at (veloFeedbackNode)  [shift={( 0.007, -0.05 )}] {\Large \textbullet}
      node at (accFeedbackNode)   [shift={( 0.007, -0.05 )}] {\Large \textbullet}
      node at (feedForward)       [shift={( 0.007, -0.05 )}] {\Large \textbullet}
      node at (frictionWheel)     [shift={( 1.025, -0.04 )}] {\Large \textbullet}
    ;

    %-- Summation signs --%
      \draw%--------------------------------------------------------------
      node at (sum1) [right = -6.6mm, below = .6mm] {$-$}
      node at (sum1) [right = -3mm, below = 3.9mm]  {$+$} 
      node at (sum2) [right = -6.6mm, below = .6mm] {$+$}
      node at (sum2) [right = -3mm, below = 3.9mm]  {$+$}
      node at (sum3) [right = -6.6mm, below = .6mm] {$+$}
      node at (sum3) [right = -3mm, below = 3.9mm]  {$-$}
      node at (sum4) [right = -6.6mm, below = .6mm] {$+$}
      node at (sum4) [right = -3mm, below = 3.9mm]  {$-$}
      node at (sum5) [right = -6.6mm, below = .6mm] {$+$}
      node at (sum5) [right = -3mm, below = 3.9mm]  {$-$}
    ;

  \end{tikzpicture}
	\caption{Block diagram of the system}
	\label{cubliSimulink}
\end{figure}

Substituting all the constants of \eqref{2ndCubliTransferFunction} with the parameters of the real model results in the final transfer function of the system.
%
\begin{flalign}
	\eq{G(s)}{\frac{-211.9 \cdot s}{s^3 + 1.32 \cdot s^2 - 98.98 \cdot s - 2.803}} &\nonumber\\
	\label{RealCubliTransferFunction}	
\end{flalign}
%
Using \eqref{RealCubliTransferFunction} it is possible to simulate the response of the system to a step input and compare it with the response of the simulation of the block diagram. This is done to verify that the block diagram is in fact showing the system described in \eqref{RealCubliTransferFunction}, as seen in \figref{stepComparison}.
%
\begin{figure}[H] 
	\centering 
	\includegraphics[scale=0.55]{figures/stepComparison}
	\caption{Step response comparison between the transfer function from \eqref{RealCubliTransferFunction} and the block diagram from \figref{cubliSimulink}. The conclusion form this is the blockdiagram of the system was made correctly}
	\label{stepComparison}
\end{figure}
%
In \figref{LinearizedVSNonlinear}, the effect of the linearization is apparent. In the simulation the frame is placed in upright position very slightly off \si{0\ rad}. The small deviation from \si{0\ rad} is applied in the last plant feedback, see \figref{cubliSimulink}. If, in the simulation, the model is started out at exactly \si{0\ rad}, it will balance in upright position.
\Figref{LinearizedVSNonlinear} shows the behavior around the frame's pivot point and does not include the platform itself. For this reason, the simulation allows for the frame to fall down and act as a normal pendulum. The nonlinear model shows how the pendulum dampens around its natural equilibrium point, while the linear model keeps increasing the angle.
However, in reality the platform blocks the frame's path, and so, it can only ever turn \si{45 ^\circ} to either side, that is, \si{\frac{\pi}{4}\ rad \approx 0,79\ rad}.
\begin{figure}[H] 
	\centering 
	\includegraphics[scale=0.55]{figures/LinearizedVSNonlinear}
	\caption{Simulation of the linearized model compared to the nonlinear model}
	\label{LinearizedVSNonlinear}
\end{figure}
From \figref{LinearizedVSNonlinear}, the linearized model is considered a good approximation of the system's behavior in the operational region of \si{\pm 0,79\ rad}. Hence the linearized model is used both for further analysis of the system and for controller design.