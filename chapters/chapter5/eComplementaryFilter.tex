\subsection{Complementary filter}
The complementary filter is used to combine the measurement data from the accelerometer and gyros in the two IMUs mounted on the Cubli frame. 
Based on the data from \appref{IMUMeasurementsAppendix} it can be seen that the accelerometer can detect changes in the angle, but has problems with measurement noise when the change is large. The problem with the data from the gyro is an accumulating error because of the integration done to convert angular velocity into an usable angle.
Based on ... \fxnote{why did we choose an complementary filter?}

The two angle measurements of the IMU are sent through a filter and then summed in order to get an angle of the Cubli frame.
%Data from the accelerometer is used to calculate an angle of the Cubli frame.
%The gyroscope measurement is integrated to get the angle of the Cubli frame. Both measurements are done to find the change on the axis of movement of the Cubli.

 
\begin{figure}[H] 
	\centering
	\includegraphics[scale=0.08]{figures/tempComplementaryFilter}
	\caption{Sketch of the structure of the complementary filter. Measurements from the accelerometer are calculated into an angle on the axis of movement. Data from gyro is used to find the angular velocity on the axis of movement}
	\label{blockComplementaryFilter}
\end{figure}
\fxnote{Make a proper sketch of the complementary filter, B}
This is done to counteract drift of the gyroscope and error of the accelerometer.\fxnote{elaborate on the errors} 

\begin{figure}[H]
	\input{figures/complementaryFilterBlockdiagram.tikz}
	\centering
	\caption{Insert Description}
	\label{blockDrawingComplementaryFilter}
\end{figure}
