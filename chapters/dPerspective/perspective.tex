\chapter{Perspective}
After designing the one-side prototype, the next step is to build a full cube version. To achieve this goal some changes and extra features must be added to the current design.

The placement of the IMUs on the full Cubli is something that has to be determined. One way to do it if possible is to place just one IMU in the very center of the Cubli.\\
Alternatively one sensor could be placed in each corner of the Cubli. That would mean using 8 IMUs. This way, there is always at least one sensor very close to the point of rotation, and thus, it should be possible to get measurements with minimal disturbance from linear acceleration. A system that keeps track of the orientation of the Cubli could be implemented, so that the controller receives information from the most reliable sensor.

Each of the three reaction wheels needs to have its own controller. In this case it is not the goal to achieve an upright position, but to perform other maneuvers. This will require a redesigned controller since the chosen one is not able to get a changing reference for the internal states.

The Cubli will have to be able to jump up from a resting position. To do that the reaction wheel has to spin up. Once at the desired speed the wheel will be braked and the resulting torque will raise the Cubli. Now the system has to catch itself. In order to do that a separate controller might have to be designed that handles the jump up part, and then switches to the balancing controller for the balancing functionality.\\
For the full Cubli, since there are three wheels and three braking systems, all the controllers will need to be coordinated.
%------------------------------------------------------------------------------

%The controller designed in this project can only keep the frame upright at the equilibrium position. It will not be possible to keep the frame at an angle different than \si{0^\circ} because the wheel cannot accelerate unlimited. Changing the restrictions that are on the system might mean replacing a few components like the motor control board. Giving the motor a higher speed ceiling allows the system to accelerate longer or accelerate faster to generate more torque. The longer the wheel can keep accelerating the longer will be able to keep the frame a position different from \si{0^\circ}.\\
%------------------------------------------------------------------------------
%The Inertia of the reaction wheel could be changed. this could be done by changing its size and/or its mass. Assuming the same torque is wanted a larger inertia means a slower acceleration is needed to generate the torque. The larger inertia also means slower response time. A wheel with smaller inertia would have to accelerate faster to generate the same torque, but it would also have a faster response time. \fxnote{somebody double check my assumption about physics} The inertia of the wheel could be something that could be optimized, depending on what is needed for the rest of the system. More torque or faster response time.
%------------------------------------------------------------------------------
%In the state space controller a set of poles were selected looking at the response they showed in two simulations, \figref{catchingStateSpace} and \figref{catchingStateSpaceWheel}. Using different pole combinations results in different overshoots and settling time, so another pole combination could be selected if a different behavior of the Cubli was desired. The choice is whether a small overshoot is desired or if the frame is to reach \SI{0}{rad} faster.

%One of the next steps will be to build the full Cubli, since the control now can be done using only components attached to the frame and is not dependent on any fixed component to find its angle. The full cube design needs to include the control of three reaction wheels to control each of the directions that Cubli can move in. 

%The Cubli will have to be able to jump up from a resting position. To do that the reaction wheel has to spin up. Once at the desired speed the wheel will be braked and the resulting torque will raise the Cubli. Now the system has to catch itself. In order to do that a separate controller might have to be designed that handles the jump up part, and then switches to the balancing controller for the position keeping part.\\
%For the full Cubli there will be 3 brakes, one on each wheel, that have to be coordinated to raise the cube up to a desired position.
%This new configuration may also need the system to be able to jump up from resting position and then the brake system must be taken into account for each of the wheels. 

%It is possible that the controller no longer needs to balance each of them in upright position but to do another kind of maneuvers, which will require a new controller since the chosen one is not able to get a reference for the internal states different from zero.

%The placement of the IMU on the full Cubli is something that has to be determined. One way to do it if possible is to place the IMU in the very center of the Cubli. This would give the same disturbance regardless of which angle has to be measured.\\
%Alternatively one sensor could be placed in each corner of the Cubli. That would mean using 4 IMU's. This way there is always at least one sensor very close the point of rotation, and thus it should be possible to get measurements with minimal disturbance from linear acceleration. A system could be implemented that keeps track of the orientation of the Cubli, so that the most reliable sensor always is selected as the one the controller trusts more. 


%- Jump up\\
%- 3D Cube\\
%- Deep more in other possibilities of controllers\\
%- May not need to balance, but to do another maneuvers -----> other controller -----> same model\\
%- Show and advertise control engineering \\
%- Sensor placement on the 3D cube.\\