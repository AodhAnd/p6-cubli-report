\section{Linearization of the Model}
Now that a model of the Cubli frame is put forth in \eqref{FrameEqFinal}, it is apparent that the system is nonlinear due to the term including \si{sin(\theta_F)}. In order to proceed with a simulation and controller design, it is convenient to first linearize the model. This is done by use of a Taylor series approximation.

Based on \eqref{FrameEq4} the system is described in an operating point, around which it varies with \si{\Delta \theta_F}.
%
\begin{flalign}
	\si{(J_F+m_w \cdot {l_w}^{2}) (\ddot{\theta}_F + \Delta \ddot{\theta}_F )} &= \si{- B_F \cdot (\dot{\theta}_F + \Delta \dot{\theta}_F) }   \nonumber\\
	&\ \ \ \ \si{+ (m_F \cdot l_F + m_w \cdot l_w) \cdot g \cdot sin(\theta_F + \Delta \theta_F)} \nonumber\\
	&\ \ \ \ \si{- \tau_M + B_w \cdot \dot{\theta}_w}  \unit{N \cdot m}\\
	\eq{(J_F+m_w \cdot {l_w}^{2}) (\ddot{\theta}_F + \Delta \ddot{\theta}_F )}{ f( (\dot{\theta}_F + \Delta \dot{\theta}_F), \ (\theta_F + \Delta \theta_F), \ (\tau_m + \Delta \tau_m),\  (\dot{\theta}_w + \Delta \dot{\theta}_w) ) } \unit{N \cdot m}
\label{FrameEq4OperatingPoint}
\end{flalign}
%
The operating point is chosen to be \si{\theta_F = 0}, which corresponds to the frame being in upright position, see \figref{cubliMechanical}. Taking this into account and applying the Taylor series approximation yields the following.
%
\begin{flalign}
	\si{(J_F+m_w \cdot {l_w}^{2}) \Delta \ddot{\theta}_F } &= \cancelto{0}{\si{f( \dot{\theta}_F, \ \theta_F, \ \tau_m,\ \ddot{\theta}_w )}}   \nonumber\\
	&\ \ \ \ \si{+ \frac{\partial}{\partial \dot{\theta}_F} f\cdot \Delta \dot{\theta}_F + \frac{\partial}{\partial \theta_F} f\cdot \Delta \theta_F + \frac{\partial}{\partial \tau_m} f\cdot \Delta \tau_m + \frac{\partial}{\partial \dot{\theta}_w} f\cdot \Delta \dot{\theta}_w } \unit{N \cdot m}
\label{FrameEq4OperatingPointZero}
\end{flalign}

All the higher order derivatives are discarded due to its negligible impact on the system.
%
\begin{flalign}
	\si{(J_F+m_w \cdot {l_w}^{2}) \Delta \ddot{\theta}_F } &= \si{-B_F \Delta \dot{\theta}_F +  ( m_F \cdot l_F + m_w \cdot l_w ) g \cdot} \cancelto{1}{\si{ \rule{0cm}{.4cm} cos(\theta_F)}} \si{\Delta \theta_F} \where{\theta_F = 0} \nonumber\\
	&\ \ \ \ \si{- \Delta \tau_m + B_w \Delta \dot{\theta}_w } \unit{N \cdot m}\\
	\eq{(J_F+m_w \cdot {l_w}^{2}) \Delta \ddot{\theta}_F }{-B_F \Delta \dot{\theta}_F +  ( m_F \cdot l_F + m_w \cdot l_w ) g \cdot \Delta \theta_F - \Delta \tau_m + B_w \Delta \dot{\theta}_w } \unit{N \cdot m}
\label{FrameEq4TaylerApprox}
\end{flalign}
%
\Eqref{FrameEq4TaylerApprox} shows the final linearized model.