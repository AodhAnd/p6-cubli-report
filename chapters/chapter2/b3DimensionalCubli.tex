\section{3 Dimensional Cubli}\label{threeDCubli}
% - Mechanical description of 3 dimensional cubli (focus: control)
%     - jump up on side
%     - jump up on corner 
%     - balancing on side
%     - balancing on corner
% - What have other cubli-makers achieved
%     - balance deviation (if moved x degrees away from equilibrium
%                          point - for how big x can it still find
%                          its equilibrium again)
%     - corner balance (duration?)
%     - general on stability
%         - Different slopes
%     - moving (rolling)
%     - spin
The full cubli setup is a 3 dimensional cube that consists of a metal frame. Inside the frame are mounted 3 reaction wheels which are driven by a dc motor and each wheel can be stopped with a brake. 
Additionally there are mounted different kind of sensors inside the frame, that are used to determine the position of the Cubli.\fxnote{Look in source material what kind of sensors the cubli is using}


As mentioned, the Cubli balances with the help of reaction wheels that spin up or down. Jumping up from a resting position is done by braking one or more of the reaction wheels, which overcomes the gravity and inertia keeping the Cubli down.
By controlling the direction of the jump-up motion and the falling direction, the Cubli can move around its body. At this point, the Cubli can be considered a cube robot.\\