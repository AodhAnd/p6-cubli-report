\section{Three Dimensional Cubli}\label{threeDCubli}
% - Mechanical description of 3 dimensional cubli (focus: control)
%     - jump up on side
%     - jump up on corner 
%     - balancing on side
%     - balancing on corner
% - What have other cubli-makers achieved
%     - balance deviation (if moved x degrees away from equilibrium
%                          point - for how big x can it still find
%                          its equilibrium again)
%     - corner balance (duration?)
%     - general on stability
%         - Different slopes
%     - moving (rolling)
%     - spin

The full Cubli setup is a 3 dimensional cube, whose structure is a metallic frame. Inside the frame, three reaction wheels are mounted and driven by DC motors and a brake system. 

Additionally there are different kind of sensors inside the frame, that are used to determine the position of the Cubli.\fxnote{Look in source material what kind of sensors the cubli is using}

The system is controlled by the use of reaction wheels, which make it able to balance on a side or a corner and to jump up from rest position when braking the wheels. It is also possible to make it move in one direction, if the control of the wheels is combined.% At this point, the Cubli can be considered a cube robot.\\