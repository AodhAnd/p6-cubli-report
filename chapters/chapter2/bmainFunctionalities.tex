\section{Desired Functionalities}\label{sec:mainFunctionalities}
In this section, a non-exhaustive list of capabilities a Cubli prototype should have, is put forth.

A simple design method, for the Cubli to stand on one of its corners, is to make it jump to this position in two steps. First, it should be made to stand on one of its edges before going to a balancing position on a corner\fxnote{Add ref}.

Therefore, as a basic ability, the prototype should be able to jump onto any one of its edges and balance.
By rotating one of its reaction wheels to a determined and controlled velocity and braking it suddenly, it should be able to raise the cube to an unstable position on an edge. When the jump up controller has raised the Cubli an other controller must catch it around equilibrium position.\\
This second controller is essential to allow the prototype to keep balancing. It should react fast enough when put into action so that the Cubli does not fall again. With some more considerations regarding this controller's robustness, it should be possible to change the inclination of the surface under it to a certain extent.\\
In a similar fashion, the second step of the process consists of speeding up wheels and braking them one more time to raise the Cubli to one of its corners. The latter should finally be caught by a controller able to maintain it in this unstable equilibrium.

Once standing on a corner, a spinning functionality should allow the prototype to turn around itself. This change in orientation also permits the Cubli to move in its environment by falling and raising repeatedly towards the desired direction.

With a cube, communication and power supply as wired connections are not practical if the prototype has to move around. To ensure a complete autonomy, the prototype should be remotely controllable, i.e. it should be possible to ask it to run pre-defined routines (stand on an edge or a corner, spin, make a full turn, etc.) from a distant computer.\\
Moreover, it should be self-sustained by an internal battery able to run the main computer, the actuators and the sensors for a reasonable amount of time.

All these functionalities are potential features to implement. The next section sets some limits as to what is actually to be achieved for this project.