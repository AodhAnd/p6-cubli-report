\chapter{System Description}
One-frame Cubli is a non-linear unstable system which is capable of laying on one of its corners as long as it is properly controlled, using  the inertia created in the reaction wheel present in the setup.

This system can be divided into the following parts: \fxnote{insert correct part names} \fxnote{Include picture of the real model, mark all the parts in the picture}

\begin{table}[H]
	\begin{tabular}{|l|p{4.5cm}|}
		\hline %-----------------------------------------------------------------------------------
		\textbf{No.} &\textbf{Part} 			\\
		\hline %-----------------------------------------------------------------------------------
		1            & Frame           			\\
		\hline %-----------------------------------------------------------------------------------
		2            & Reaction wheel      		\\
		\hline %-----------------------------------------------------------------------------------
		3            & Controller (BeagleBone)  \\
		\hline %-----------------------------------------------------------------------------------
		4            & Potentiometer			\\
		\hline %-----------------------------------------------------------------------------------
		5            &  IMU          			\\
		\hline %-----------------------------------------------------------------------------------
		6            & Brushless DC-motor   	\\
		\hline %-----------------------------------------------------------------------------------
		7            & Motor control board     	\\
		\hline %-----------------------------------------------------------------------------------
		8            & Jump brake		    	\\
		\hline %-----------------------------------------------------------------------------------
	\end{tabular}
	\caption{Main parts of the Cubli setup \label{TableAAUCubliComponent}}
\end{table}


\section{Frame}
The frame is a metallic \fxnote{find composition of frame if possible} 15x15cm square with two cross connections between opposing corners, that composes the body of the Cubli. As it can only be balanced in one direction, it is attached to the base on one of its vertexes to avoid it from falling in any of the other directions.

Another important aspect is that all the electronics needed and the motor are attached to the frame. This has to be taken into account when finding the model, as its moment of inertia, center of gravity and mass change.


\section{Reaction Wheel}
The reaction wheel is made of metal? \fxnote{find composition of wheel if possible} with most of the mass in a ring at its outer edge and two cross connections through its center.

It is the element of Cubli that is responsible of the movement of the frame. This can be achieved since the rotation of the wheel, which is coupled to the axis of the motor thought its center of rotation, creates an opposing torque on the body that moves it.


\section{Controller}
The controller used on this system is a BeagleBone, which is in charge of managing the data from the motor and the sensors and calculating the required control action.


\section{Potentiometer}
The potentiometer is placed at the corner of the frame that is fixed to an axis and it can be used to measure the actual position of the frame.

However its use is restricted to the one-dimensional Cubli, as it has to be referred to the base, which is not present in the 3D case.


\section{IMU}


\section{Brushless DC-Motor}


\section{Motor Control Board}


\section{Jump Brake}
The last important part included on the system is the braking mechanism, which is used for making the frame go from rest to vertical position. 

To perform this task the motor spins up the reaction wheel and when it has enough kinetic energy the braking mechanism suddenly brake it. This makes that all the energy of the wheel is then transfer to the frame, which reaches the vertical position.
