\chapter{Design Considerations}\label{chap:designConsiderations}
% - Mechanical description of 3 dimensional cubli (focus: control)
%     - jump up on side
%     - jump up on corner 
%     - balancing on side
%     - balancing on corner
% - What have other cubli-makers achieved
%     - balance deviation (if moved x degrees away from equilibrium
%                          point - for how big x can it still find
%                          its equilibrium again)
%     - corner balance (duration?)
%     - general on stability
%         - Different slopes
%     - moving (rolling)
%     - spin

From the introduced applications both in modular robot design as well as planetary or asteroid exploration, a high controllability Cubli design is desired.
In a realization of a full functioning Cubli, different considerations regarding design and overall functionalities must be taken into account. Later in this chapter, limitations are considered regarding the features to implement in this project.

% The setup for this project is given by the university, which means that some considerations must be taken into account when modelling the system and designing a controller to balance it.

% The full Cubli setup is a 3 dimensional cube, whose structure is a metallic frame. Inside the frame, three reaction wheels are mounted and driven by DC motors and a brake system and additionally there are different kind of sensors inside the frame, that are used to determine the position of the Cubli.
% %\fxnote{Look in source material what kind of sensors the cubli is using}

% The system is controlled by the use of reaction wheels, which make it able to balance on a side or a corner and to jump up from rest position when braking the wheels. It is also possible to make it move in one direction, if the control of the wheels is combined.% At this point, the Cubli can be considered a cube robot.\\

% On the other hand, the working setup that exists at Aalborg University (AAU) is composed by one of the sides of the 3D cube. One of the most important characteristics is that one of its corners is attached to a base plate. This feature limits the number of movements the Cubli can do, as it can only be balanced in one direction along a vertical plane.

% Since all the hardware is already built, the goal of the project is focused on deriving the model of the system, simulating it to compare it with the real Cubli and, moreover, on designing a controller which is able to balance it in the equilibrium position.

%At Aalborg University (AAU), there exists a working setup of a one-dimensional Cubli. The overall goals of this semester are to make a model of a system, to simulate this model and then to design and implement a controller for that system. Students on this semester are encouraged to work with pre-made setups, since the focus is rather on the control engineering than the hardware solution.\fxnote{moved the section directly from introduction. Will need a top to catch the previous chapter}

\begin{figure}[H]
	  \begin{tikzpicture}[ auto,
thick,                         %<--setting line style
node distance=2cm,             %<--setting default node distance
scale=1.0,                     %<--|these two scale the whole thing
every node/.style={scale=1.0}, %<  |(always change both)
>=triangle 45 ]                %<--sets the arrowtype
  ]
  
  %-- Blocks creation --%
  \draw
	node[shape=coordinate][](ref) at (0,0){\si{\theta_F}}			% start of reference
 
	node(sum1) at (2,0) [sum] {$\sum$}
 
	node(D2) at (4,0) [block] {$D2$}
	
	node(sum2) at (6,0) [sum] {$\sum$}
	
	node(D1) at (8,0) [block] {$D1$}
	
	node(G1) at (10,0) [block] {$G1$}
	
	node[shape=coordinate][](velFeed) at (11,0){omega\_f}
	
	node(G2) at (12,0) [block] {$G2$}
	
	node[shape=coordinate][](feed) at (13,0){feed}
	
	node[shape=coordinate][](angleFeed) at (14,0){theta\_f}
	
	node[shape=coordinate][](out) at (15,0){theta\_f}
	
	% - feedback nodes
	
	node[shape=coordinate][](feed1) at (9,-1.5){feed1}
	
	node[shape=coordinate][](feed2) at (9,-2){feed2}
	
	node[shape=coordinate][](feed3) at (12,-1){feed3}
	
  ;
  
  \draw[->](ref) -- node {\si{\theta_F}} (sum1);
  
  \draw[->](sum1) -- node {} (D2);
  
  \draw[->](D2) -- node {} (sum2);
  
  \draw[->](sum2) -- node {} (D1);
  
  \draw[->](D1) -- node {} (G1);
  
  \draw[->](G1) -- node {\si{\dot{\theta}_F}} (G2);
  
  \draw[->](G2) -- node {\si{\theta_F}} (out);
  
  % - drawing feedback lines
  
  \draw[-](velFeed) |- node {} (feed1);
  
  \draw[-](angleFeed) |- node {} (feed2);
  
  \draw[-](feed1) -| node {} (sum2);
   
  \draw[-](feed2) -| node {} (sum1);
  
  \draw[dashed](feed) |- node {} (feed3);
  
  \draw[dashed, ->](feed3) -| node {} (G1);
  
  % - adding + and - at the sum nodes
    \draw
  node at (sum1) [right = -6.6mm, below = .6mm] {$+$}
  node at (sum1) [right = -3mm, below = 3.9mm]  {$-$} 
  
  node at (sum2) [right = -6.6mm, below = .6mm] {$+$}
  node at (sum2) [right = -3mm, below = 3.9mm]  {$-$} 
  ;
 
  
  \end{tikzpicture}
	\caption{Block diagram of a classical control setup}
	\label{classicalControl}
\end{figure}