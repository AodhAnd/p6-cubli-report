\section{Verification of the Model}

To verify the model in simulation \eqref{FrameEq4TaylerApprox} is transformed into the Laplace domain, after which a transfer function of the system is derived. The proceeding equations are valid only around the operating point, and so for better overview, in the following \si{\Delta \theta_F = \theta_F}.
%
\begin{flalign}
	\eq{(J_F+m_w \cdot {l_w}^{2}) \cdot \theta_F \cdot s^2}{-B_F \theta_F\cdot s +  ( m_F \cdot l_F + m_w \cdot l_w ) g \cdot \theta_F - \tau_m + B_w \theta_w\cdot s } \unit{N \cdot m}
\label{LaplaceOfLinearizedModel}
\end{flalign}
%
The angle of the reaction wheel, \si{\theta_w}, still features in \eqref{LaplaceOfLinearizedModel}. It is desireable to have only one input, \si{\tau_m}, and one output, \si{\theta_F}. To achieve this, \eqref{WheelRotEq2} is transformed into the Laplace domain and solved for \si{\theta_w}.
%
\begin{flalign}
	\eq{\theta_w\cdot s^2} {\frac{\tau_m - B_w \theta_w\cdot s}{J_w} - \theta_F\cdot s^2}   &\\
	\eq{\theta_w} {\frac{ -J_w \theta_F \cdot s^2 + \tau_m }{ J_w \cdot s^2 + B_w \cdot s }}&
\label{WheelRotEq2Laplace}
\end{flalign}
%
\Eqref{WheelRotEq2Laplace} is now substituted for \si{\theta_w} in \eqref{LaplaceOfLinearizedModel}, and the transfer function is of the system is derived.
%
\begin{flalign}
	\si{(J_F+m_w \cdot {l_w}^{2}) \cdot \theta_F \cdot s^2} &= \si{-B_F \theta_F\cdot s +  ( m_F \cdot l_F + m_w \cdot l_w ) g \cdot \theta_F - \tau_m} &\nonumber\\
	&\ \ \ \ \si{+ B_w ( \frac{ -J_w \theta_F \cdot s^2 + \tau_m }{ J_w \cdot s^2 + B_w \cdot s } )\cdot s }&
\label{CubliTransferFunction}
\end{flalign}