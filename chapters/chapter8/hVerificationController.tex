\section{Controller Analysis}\label{ssec:ControllerVerification}
A further analysis can be done to the controller, both continuous and discrete, and also to the real response of the system once it is implemented.

The first step is to simulate both the continuous and discrete controller with the model of the system and analyse the behavior of the whole closed loop system.

This is done not only to see the behavior of the designed controller but also to verify that the discretized controller matches the original continuous one. 

With a constant reference of 0 rad and a disturbance in the form of a torque applied to the frame of \si{0,55 Nm}, the response is the one shown in \figref{discreteVsContinuousOutputController} and \figref{discreteVsContinuousSimulation}.
%
\begin{minipage}{0.45\linewidth}
	\begin{figure}[H]
      \includegraphics[scale=.53]{figures/torqueComp}
      \captionsetup{justification=centering}
      \captionof{figure}{Controller's output (torque) response in the control loop with the continuous (blue) and discrete (red) controllers}
      \label{discreteVsContinuousOutputController}
    \end{figure}\vspace{-5mm}
\end{minipage}
\hspace{0.03\linewidth}
\begin{minipage}{0.45\linewidth}
    \begin{figure}[H]\vspace{-4mm}
      \includegraphics[scale=.53]{figures/positionComp}
      \captionsetup{justification=centering}
      \captionof{figure}{Closed loop response of the continuous (blue) and discrete (red) controllers}
      \label{discreteVsContinuousSimulation}
    \end{figure}\vspace{-5mm}
\end{minipage}

Both controllers seem to have a good behavior and both reach the desired final position. However it is important to look also to the velocity of the wheel, since it has been assumed to be 0 in the equilibrium position.
%
\begin{figure}[H]\vspace{-4mm}
	\centering
	\includegraphics[scale=.53]{figures/wheelComp}
	\captionof{figure}{Angular velocity of the wheel}
   \label{fig:discreteVsContinuousWheel}
\end{figure}\vspace{-5mm}

In both cases the velocity is different from 0 in the steady-state

