\section{Analysis of the Continuous Controller}\label{analysisController}
A first analysis of the controller can be done through the resultant Nyquist plot.


Another step in the analysis is by means of a simulation, usng the block diagram of the plant.

%The resultant system can be seen in figures \figref{}.

When a reference of 0 rad is given to the system, which has a tiny deviation from the equilibrium position in the initial condition, the behavior is the one shown in \figref{TorqueResponse} and \figref{PositionResponse}.

%\begin{minipage}{\linewidth}
%	\begin{minipage}{0.45\linewidth}
%		\begin{figure}[H]
%			\includegraphics[scale=.56]{figures/TorqueResponse}
%			\centering
%			\captionsetup{justification=centering}
%			\captionof{figure}{Torque needed at the motor}
%			\label{TorqueResponse}
%		\end{figure}
%	\end{minipage}
%	\hspace{0.03\linewidth}
%	\begin{minipage}{0.45\linewidth}
%		\begin{figure}[H]
%			\includegraphics[scale=.56]{figures/PositionResponse}
%			\centering
%			\captionsetup{justification=centering}
%			\captionof{figure}{Angular position of the system}
%			\label{PositionResponse}
%		\end{figure}
%	\end{minipage}
%\end{minipage}

As it can be seen, the controller is capable of balancing the Cubli at position 0 rad with a control action (torque) which can only take values from -0.134 to 0.134 as the real actuator does.