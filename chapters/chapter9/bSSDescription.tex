\section{State-Space Representation of the System}\label{sec:SSDescription}
The first step to give a representation of the system in state-space is to choose the input, output and state variables. In this case the input is the torque from the motor (\si{\tau_m}), the state variables are the position of the frame and the velocities of frame and wheel (\si{\theta_F,\ \dot{\theta}_F,\ \dot{\theta}_w}), and the outputs are the ones to control (\si{\theta_F,\ \dot{\theta}_w}).
%
\begin{minipage}{0.32\linewidth}
	\begin{flalign}
		x = 
		\begin{bmatrix}
			\theta_F \\
			\dot{\theta}_F \\ 
			\dot{\theta}_w \\
		\end{bmatrix}	\nonumber
		\label{xVector}
	\end{flalign}  
\end{minipage}\hfill
%\hspace{0.03\linewidth}
\begin{minipage}{0.32\linewidth}
	\begin{flalign}
		y = 
		\begin{bmatrix}
			\theta_F \\
			\dot{\theta}_w \\
		\end{bmatrix}	\nonumber
		\label{yVector}
	\end{flalign}
\end{minipage}\hfill
%\hspace{0.03\linewidth}
\begin{minipage}{0.32\linewidth}
	\begin{flalign}
		u= 
		\begin{bmatrix}
			\tau_m\\
		\end{bmatrix}
		\label{uVector}
	\end{flalign}
\end{minipage}\hfill

The relationship between them is  given in the form of first-order differential equations:
\begin{flalign}
	\eq{\dot{x}}{f(x,u)}
	\label{xDotDiffEq} 
\end{flalign}
\begin{flalign}
	\eq{y}{g(x,u)} 
	\label{yDiffEq} 
\end{flalign}
%
In this case f and g can be derived from \eqref{finalWheel} and \eqref{finalFrame} and they are non-linear functions since there exist a sinusoidal term in both.

However, it has been shown that a linear approximation near the equilibrium point is possible. Then, the system can be describe in the form of linear state-space equations as seen in \eqref{xDotLinear} and \eqref{yLinear}.
%
\begin{flalign}
	\eq{\dot{x}(t)}{A \cdot x(t) + B \cdot u(t)}
	\label{xDotLinear} 
\end{flalign}
\begin{flalign}
	\eq{y(t)}{C \cdot x(t) + D \cdot u(t)}
	\label{yLinear} 
\end{flalign}
%
\hspace{6mm} Where:\\
\begin{tabular}{ p{1cm} l l l}
	& A			& is the \si{3x3}  state matrix \\                       
	& B			& is the \si{3x1}  input matrix \\ 
	& C			& is the \si{2x3}  output matrix \\ 
	& D			& is the \si{2x1}  feedforward matrix \\ 
\end{tabular} 
\\
The state-space description can be seen also in the form of a block diagram like the one from \figref{SSBlocks}.
%
\begin{figure}[H]
	\begin{tikzpicture}[ auto,
                       thick,                         %<--setting line style
                       node distance=1.5cm,             %<--setting default node distance
                       scale=1,                     %<--|these two scale the whole thing
                       every node/.style={scale=1}, %<  |(always change both)
                       >=triangle 45 ]                %<--sets the arrowtype
    
    \draw%-----------------------------------------------------------------------------------------
    	%Drawing Input/Output:
    	node[shape=coordinate][](input1) at (0,0){}
    	node[shape=coordinate][](output1) at (9.5,0){}
     	%Drawing the Equation Blocks:   	
      	node(A) at (4.5,-1.5) [block] {A} 
     	node(B) at (1.5,0) [block] {B}
     	node(C) at (6.5,0) [block] {C}
      	node(D) at (4.5,1.5) [block] {D}  
	    node(int) at (4.5,0) [block] {\si{\int}}  
    	%Drawing the Sumation Blocks:	    	 	
    	node(sum1) [sum, right of = B] {\si{\sum}}
    	node(sum2) [sum, right of = C] {\si{\sum}}
    	%Drawing the Feedback/Feedforward Nodes:    	
    	node[shape=coordinate][](FeedforwardNode) at (0.75,0){}
    	node[shape=coordinate][](FeedbackNode) at (5.5,0){}  	
    	     
    ;%---------------------------------------------------------------------------------------------
   
    %Joining the Blocks
  	\draw[->](input1) -- node {u}(B);
  	\draw[->](B) -- node {}(sum1);
  	\draw[->](sum1) -- node {\si{\dot x}}(int);  	
  	\draw[->](int) -- node {x}(C);
  	\draw[->](C) -- node {}(sum2);  	
  	\draw[->](sum2) -- node {y}(output1);
  	
  	\draw[->](FeedforwardNode) |- node{} (D);
  	\draw[->](D) -| node{} (sum2);

  	\draw[-] (FeedbackNode) |- (A);
  	\draw[->] (A)   -| (sum1);

    %Drawing node(s) with \textbullet
    \draw%--------------------------------------------------------------
      node at (input1)  [shift={(-0.08, -0.02 )}] {\large \textbullet}
    	% node at (output1) [shift={( 0.008, -0.02 )}] {\textbullet}
    ;%------------------------------------------------------------------
  \end{tikzpicture}
	\centering
	\caption{Block diagram of the state-space representation}
\end{figure} \label{SSBlocks}
%
This matrices can be obtained from the linearized equations of the system from \ref{sec:linearization}, given the final system description as \eqref{xDotSS} and \eqref{ySS}.

\begin{equation}  \label{xDotSS}
	\dot{x}(t) = 
	\begin{bmatrix}
		0 & 1 & 0 \\
		\frac{(m_F \cdot l_F + m_w \cdot l_w) \cdot g}{J_F + m_w \cdot {l_w}^2} & -\frac{B_F}{J_F + m_w \cdot {l_w}^2} & \frac{B_w}{J_F + m_w \cdot {l_w}^2} \\
		- \frac{(m_F \cdot l_F + m_w \cdot l_w) \cdot g}{J_F+m_w \cdot {l_w}^{2}} & \frac{B_F}{J_F+m_w \cdot {l_w}^{2}} & \frac{(J_w+J_F+{l_w}^{2} \cdot m_w) \cdot B_w}{J_w \cdot (J_F+m_w \cdot  {l_w}^2)} \\
	\end{bmatrix}
	\cdot x(t) +
	\begin{bmatrix}
		0 \\
		- \frac{1}{J_F + m_w \cdot {l_w}^2} \\
		\frac{J_w+J_F+m_w \cdot {l_w}^{2}}{J_w \cdot (J_F+m_w \cdot {l_w}^{2})} \\
	\end{bmatrix}
	\cdot u(t)
\end{equation}
\begin{equation} \label{ySS}
	y(t) = 
	\begin{bmatrix}
		1 & 0 & 0 \\
		0 & 0 & 1 \\
	\end{bmatrix}
	\cdot x(t) +
	\begin{bmatrix}
		0 \\
		0 \\
	\end{bmatrix}
\cdot u(t)
\end{equation}

