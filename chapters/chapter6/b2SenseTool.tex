\section{Parameter Estimation using Optimization}
In this section two methods for optimization will be investigated. The bases of implementation is in this case a Matlab script which is fed test data along with the model simulation, and then given the task to fit the model output to the test data by adjusting one or more parameters in the model. In this case the model representation supplied to the script is a Simulink model, which can then be run by the script whenever needed and the script can modify the parameters to be adjusted. The process is iterative in most cases.

\subsection{The Optimization Problem}
The data provided is taken from an initial value test of the Cubli hanging down like a pendulum, see \appref{impulseResponseAppendix}. Furthermore the nonlinear model is used to accurately describe the oscillatory behavior of the pendulum. The model is modified, as seen in \figref{blockDiagramSenseTool}, such that it describes the system as a regular pendulum without the dynamics of the reaction wheel, in order to match the test conditions under which the data was extracted. In order to minimize the difference between the data points measured in test and the output of the model a function to describe such a relationship is needed. The performance function used to describe goodness of the fit, is a mean square error function.
%
\begin{flalign}
	\eq{\vec{P}(\vec{\theta})} {\frac{1}{2N}\sum_{k = 1}^{N} \left(\vec{y}(kT) - \vec{y_m}(kT, \vec{\theta})\right)^2 } &
\label{performanceFunction}
\end{flalign}
%
\hspace{6mm} Where:\\
\begin{tabular}{ p{1cm} l l l}
& \si{\vec{\theta}}   & is the parameter(s) to be adjusted                  & \\
& \si{N}              & is the degrees of freedom for each parameter        & \\
& \si{k}              & is the sample indexes, \si{k=1,\ 2,} ...\si{,\ N}   & \\
& \si{T}              & is the sampling time                                & \\
& \si{\vec{y}}        & is the test measurement output vector               & \\
& \si{\vec{y_m}}      & is the model output vector                          & \\
\end{tabular}

A normal mean square error function is only divided by the degrees of freedom, \si{N}, and not \si{2N}. This is done to cancel out the factor two which arises when computing the gradient of the performance function. This only gives the function a constant offset and does not have any impact when minimizing it.

\subsection{Steepest Descent Method}
One way of solving the optimization problem is through use of the gradient. The gradient indicates in which direction the steepest descent or ascent is found in an infinitesimal surrounding of a given starting point.

For a function \si{f(x)} with a change in \si{x} of \si{\delta} the following can be obtained 
from the Taylor series.
%
\begin{flalign}
  f(x) + \Delta f(x) = f(\vec{x}+\vec{\delta}) &\approx f(x) + \vec{g}^T \vec{\delta} + \frac{1}{2} \vec{\delta}^T \vec{H}\vec{\delta} &
\label{taylorApproximation}
\end{flalign}
%
\hspace{6mm} Where:\\
\begin{tabular}{ p{1cm} l l l}
& \si{\vec{g}} 					    	   & is the gradient \si{\nabla f(x)}     & \\
& \si{\vec{H}} 					    	   & is the Hessian                       & \\
& \si{\vec{\delta}} 					   & is the change in \si{x}              & \\
\end{tabular}

%Then the change in \si{f(x)} as \si{||\vec{\delta}||_2 \rightarrow 0} can be approximated as follows.
%%
%\begin{flalign}
%  \Delta f(x) &\approx \vec{g}^T \vec{\delta} = ||\vec{g}||_2 \ ||\vec{\delta}||_2 \ cos \theta &
%\label{changeInF}
%\end{flalign}
%%
%\hspace{6mm} Where:\\
%\begin{tabular}{ p{1cm} l l l}
%& \si{\theta} 					    	   & is the angle between \si{\vec{g}} and \si{\vec{\delta}}     & \\
%\end{tabular}

In steepest descend method only the first order Taylor approximation is used, that is, the last term, \si{\frac{1}{2} \vec{\delta}^T \vec{H}\vec{\delta}} is discarded. If the derivative of the first order approximation, is set to zero, the following is obtained.
%
\begin{flalign}
  \frac{\partial}{\partial \vec{\delta}} \ f(\vec{x}+\vec{\delta}) &\approx \vec{g} = 0 &
\label{1stOrderTaylorApproximationParThetaEqZero}
\end{flalign}

That is, if the gradient of the function to be minimized is 0, a minimum or maximum exists as a candidate for a solution in this point. It follows that if standing in some point and computing the gradient in this point, then the gradient, \si{\vec{g}}, is the steepest ascend and the negative gradient, \si{-\vec{g}}, is the steepest descend. This only takes into account the immediate surroundings of the initially chosen point. A visualization of how the negative gradient points to a minimum of an arbitrary function is seen in \figref{visualizationOfGradient}.

\begin{figure}[H] 
	\centering
	\includegraphics[width=.8\textwidth]{figures/visualizationOfGradient}
	\caption{Visualization of gradient of an arbitrary function.}
	\label{visualizationOfGradient}
\end{figure}

One way of implementation is to set a step-size which decides how far in the \si{-\vec{g}} direction to go. The step-size can then be scaled in each iteration to avoid taking too large steps as shown in \figref{SteepestDescendLargeStep} and \ref{SteepestDescendSmallStep}, where \si{s^*} is the value of \si{x} which minimizes \si{f(x)}, \si{x_0} is the starting point at which \si{-\vec{g}} is computed and \si{x} is the point reached after the step.
%
\begin{minipage}{\linewidth}
	\begin{minipage}{0.45\linewidth}
		\begin{figure}[H]
			\includegraphics[scale=1.4]{figures/gradientDescendLargeStep}
			\centering
			\captionsetup{justification=centering}
			\captionof{figure}{A too large step will cause the algorithm to step over the valley, resulting in a larger value of f(x) in the \si{-\vec{g}} direction}
			\label{SteepestDescendLargeStep}
		\end{figure}
	\end{minipage}
	\hspace{0.03\linewidth}
	\begin{minipage}{0.45\linewidth}
		\begin{figure}[H]
			\includegraphics[scale=1.4]{figures/gradientDescendReducedStep}
			\centering
			\captionsetup{justification=centering}
			\captionof{figure}{By going back and choosing a smaller step, a smaller value for f(x) is obtained}
			\label{SteepestDescendSmallStep}
		\end{figure}
	\end{minipage}
\end{minipage}

The steepest descend method does find a minimum, however, it converges to the minimum rather slowly. An implementation where it is possible to directly retrieve the gradient of the function, \si{f(x)}, which is to be minimized is shown in \figref{steepestDescendEx}.

\begin{minipage}{\linewidth}
	\begin{minipage}{0.45\linewidth}
		\begin{figure}[H]
			\includegraphics[scale=.6]{figures/steepestDescendEx}
			\centering
			\captionsetup{justification=centering}
			\captionof{figure}{An example of a direct implementation of the steepest descend method. The read is where an iteration steps over the valley and the step size is reduced.}
			\label{steepestDescendEx}
		\end{figure}
	\end{minipage}
	\hspace{0.03\linewidth}
	\begin{minipage}{0.45\linewidth}
		\begin{figure}[H]
			\includegraphics[scale=.6]{figures/steepestDesendExZoom}
			\centering
			\captionsetup{justification=centering}
			\captionof{figure}{From the zoom on the convergence it is clear that many iterations (here 100) are needed using the steepest descend method.}
			\label{steepestDesendExZoom}
		\end{figure}
	\end{minipage}
\end{minipage}

From the zoom on \figref{steepestDesendExZoom} the convergence toward the minimum is clear, however it is also clear that many iterations are needed to get there.

\subsection{Newton's Method}
An other approach is called Newton's Method is also rooted in the Taylor approximation, this however method however also uses the second order term of the approximation.
%
\begin{flalign}
  f(x) + \Delta f(x) = f(\vec{x}+\vec{\delta}) &\approx f(x) + \vec{g}^T \vec{\delta} + \frac{1}{2} \vec{\delta}^T \vec{H}\vec{\delta} &
\label{taylorApproximation2ndOrder}
\end{flalign}

In this case the derivative of the approximation is set to 0, and the following is obtained.
%
\begin{flalign}
  \frac{\partial}{\partial \vec{\delta}} \ f(\vec{x}+\vec{\delta}) &\approx \vec{g} + \frac{1}{2}\ \frac{\partial}{\partial \vec{\delta}}\ \vec{H}\vec{\delta}^2 &\\
  \frac{\partial}{\partial \vec{\delta}} \ f(\vec{x}+\vec{\delta}) &\approx \vec{g} + \vec{H}\vec{\delta} = 0 &
\label{2stOrderTaylorApproximationParThetaEqZero}
\end{flalign}

Using this to find an expression for the difference in \si{x}, \si{\vec{\delta}}, yields the following.
%
\begin{flalign}
  0 &= \vec{g} + \vec{H}\vec{\delta}  &\\
  \vec{\delta} &= -\vec{H}^{-1}\vec{g} &
\label{NewtonsMethod}
\end{flalign}

This expression can then be used to choose the value of \si{\vec{\delta}} such that \si{f(x)} is minimized. An implementation where it is possible to directly retrieve the gradient and a Hessian of the function, \si{f(x)}, which is to be minimized is shown in \figref{NewtonsMethodEx}. If compared to the steepest descend method in \figref{steepestDescendEx} and \ref{steepestDesendExZoom}, where 100 iterations were used, it is clear how Newton's method converges much faster, 30 iterations, to the minimum of \si{f(x)}. 

\begin{figure}[H] 
	\centering
	\includegraphics[width=.7\textwidth]{figures/NewtonsMethodEx}
	\caption{Example of a direct implementation of Newton's method for optimization.}
	\label{NewtonsMethodEx}
\end{figure}

\subsection{Implementation of Newton's Method}
When implementing Newton's Method both the gradient and the Hessian of the performance function \eqref{performanceFunction} is needed.
%
\begin{flalign}
	\frac{\partial \vec{P}(\vec{\theta}) }{\partial \vec{\theta}} &= G(\vec{\theta}) = \frac{\partial}{\partial \vec{\theta}} \left( \frac{1}{2N}\sum_{k = 1}^{N} \left(\vec{y}(kT) - \vec{y_m}(kT, \vec{\theta})\right)^2 \right) &\\
  \eq{\vec{G}(\vec{\theta})} {- \frac{1}{N}\sum_{k = 1}^{N} \left((\vec{y}(kT) - \vec{y_m}(kT, \vec{\theta})) \ \frac{\partial  \vec{y_m}(kT, \vec{\theta})}{\partial \vec{\theta}} \right) } &
\label{gradientOfPerformanceFunction}
\end{flalign}

Since the Matlab implementation uses a Simulink model the problematic part of \eqref{gradientOfPerformanceFunction} is the derivative of the model with respect to the model parameters, \si{\frac{\partial  \vec{y_m}(kT, \vec{\theta})}{\partial \vec{\theta}}}. To solve this problem a numerical differentiation of the model is applied as shown in \autoref{AlgorithmForNummericalDiff}.

\begin{lstlisting}[ language = Matlab,
                    caption  = {Algorithm for numerical differentiation of the simulated model},
                    label    = AlgorithmForNummericalDiff ]

  %small deviation from parameters, J_f and B_f, is set
  p = 0.001;
  
  %calculating the deviation
  deltaJ_f = p*J_f; deltaB_f = p*B_f;
  
  %saving the old parameters:
  J_f_old = J_f; B_f_old = B_f;
  
  %setting deviating parameters ready for simulation
  J_f = deltaJ_f;
  
  %running the simulation again, now with deviation in J_f
  sim('CubliParameterEstimation.slx');
  
  %storring the result of the simulation
  deltaYmJf = simOut;
  
  %setting deviating parameters ready for simulation
  B_f = deltaB_f; J_f = J_f_old; %<--restoring J_f
  
  %running the simulation again, now with deviation in B_f
  sim('CubliParameterEstimation.slx');
  
  %storring the result of the simulation
  deltaYmBf = simOut;
  
  %restoring the parameters to their original value
  B_f = B_f_old;
  
  %calculating the derivatives of the model
  YmDiffBf = ( deltaYmBf - Ym )/ p;
  YmDiffJf = ( deltaYmJf - Ym )/ p;
\end{lstlisting}

Now that the model partial derivatives are found, all parts of the gradient, as represented in \eqref{gradientOfPerformanceFunction}, are known. The Hessian is also needed and can be represented as,
%
\begin{flalign}
	\vec{H}(\vec{\theta}) &= \frac{\partial^2 \vec{P}(\vec{\theta}) }{\partial \vec{\theta}\partial \vec{\theta^T}} = \frac{\partial \vec{G}(\vec{\theta}) }{\partial \vec{\theta}} &
\end{flalign}
%
which leads to the following:
\begin{flalign}
	\vec{H}(\vec{\theta}) &= \frac{1}{N}\sum_{k = 1}^{N} \left(   \frac{\partial  \vec{y_m}(kT, \vec{\theta})}{\partial \vec{\theta}} \left(\frac{\partial \vec{y_m}(kT, \vec{\theta})}{\partial \vec{\theta}} \right)^T  	  - \left(\frac{\partial^2 \vec{P}(\vec{\theta}) }{\partial \vec{\theta}\partial \vec{\theta^T}}\right) (\vec{y}(kT) - \vec{y_m}(kT, \vec{\theta}))  \right) &
\label{hessianOfPerformanceFunction}
\end{flalign}
%
To avoid the 2nd derivative of the performance function in \eqref{hessianOfPerformanceFunction} the Hessian can be approximated simply by removing this last term.
\begin{flalign}
	\vec{\widetilde{H}}(\vec{\theta}) &\triangleq \frac{1}{N}\sum_{k = 1}^{N} \left(   \frac{\partial  \vec{y_m}(kT, \vec{\theta})}{\partial \vec{\theta}} \left(\frac{\partial \vec{y_m}(kT, \vec{\theta})}{\partial \vec{\theta}} \right)^T \right) &
\label{hessianApproxOfPerformanceFunction}
\end{flalign}
%
This approximation assumes that the model is only linearly dependent on the parameters, \si{\vec{\theta}}. As the error term, \si{(\vec{y}(kT) - \vec{y_m}(kT, \vec{\theta}))}, approaches zero the approximation becomes increasingly accurate.

Running the implementation of Newton's method approximates the model parameters to \si{J_F = } and \si{B_F = }, while giving a normed root mean square error of \si{31.71 \%} as seen on \figref{ParameterEstimationNewtonCubli}. The normed RMS error is a way to determine the goodness of the fit.

\begin{figure}[H] 
	\centering
	\includegraphics[width=.8\textwidth]{figures/ParameterEstimationNewtonCubli}
	\caption{The result of implementation of Newton's method}
	\label{ParameterEstimationNewtonCubli}
\end{figure}

A lot of the background for the implementation above comes from documentation on a Matlab toolbox called Sense Tool, which is discussed in the following.

\subsection{Sense Tool}
To do this estimation, the toolbox needs input and output data from a test on the real system. In this case the data is from an initial value test of the Cubli hanging down like a pendulum, see \appref{impulseResponseAppendix} and the input is a zero-vector. A functional simulation of the model which depends on the parameters which need to be estimated must also be provided. Then it tries to minimize the error between the output data and the output of the simulation, using Gauss-Newton method.

\begin{figure}[H]
	\centering
	\includegraphics[scale=0.4]{figures/SensToolSchema}
	\caption{Schematic of Senstool}
	\label{SensToolSchema}
\end{figure}
%
The simulation of the system is given by a Simulink file, which includes the block diagram of the system without the influence of the speed of the wheel (\figref{blockDiagramSenseTool}).
%
\begin{figure}[H]
	  \begin{tikzpicture}[ auto,
                       thick,                         %<--setting line style
                       node distance=1.5cm,             %<--setting default node distance
                       scale=0.75,                     %<--|these two scale the whole thing
                       every node/.style={scale=0.62}, %<  |(always change both)
                       >/.tip={Triangle[angle=40:5pt]}
                       ]

    %-- Blocks creation --%
    \draw
      % DIRECT TERM %
      node[shape=coordinate][](input1) at (0,0){}
      node[shape=coordinate][](feedForward) at (0.5,0){}
      node(sum1) at (2.25,0) [sum]{\si{\sum}}
      node(sum2) at (3.75,0) [sum]{\si{\sum}}

      node(torque2rotacc1) at (6.85,0) [block]{\large \si{\frac{1}{J_F + J_w + m_w \cdot {l_w}^{2}}}}

      node(integration1) at (10.75,0) [block] {\large \si{\frac{1}{s}}}
      node(integration2) at (14.2,0) [block] {\large \si{\frac{1}{s}}}

      node[shape=coordinate][](output) at (15.3,0){}
      node[shape=coordinate][](veloFeedbackNode) at (11.8,0){}
      
    ;
    \draw
      % FEEDBACKS %
      node(veloFeedback) at (7,-1) [block] {\large \si{B_F}}
      node(angleFeedback) at (8,-2) [block] {\large \si{(m_F \cdot l_F + m_w \cdot l_w)g}}
    ;
    %-- Block linking --%
    % INPUT %
    \draw[-](input1)        -- node{\large \si{\tau_m(s)}}(feedForward);
    \draw[->](feedForward)  -- (sum1);

    % OUTPUT %
    \draw[-](integration2)  -- (output);
    \draw[->](output)       -- node {\large \si{\theta_{F}(s)}} (17,0);

    % DIRECT TERM %
    \draw[->] (sum1)            -- (sum2);
    \draw[->] (sum2)            -- (torque2rotacc1);
    \draw[->] (torque2rotacc1)  -- node{\large \si{\alpha_F(s)}}(integration1);
    \draw[->] (integration1)    -- node{\large \si{\omega_F(s)}}(integration2);

    % FEEDBACKS

    \draw[->] (output)           |- (angleFeedback);
    \draw[->] (angleFeedback)    -| (sum1);

    \draw[->] (veloFeedbackNode) |- (veloFeedback);
    \draw[->] (veloFeedback)     -| (sum2);

    %-- Nodes --%
    \draw%--------------------------------------------------------------
      node at (input1)            [shift={(-0.04, -0.05 )}] {\Large \textopenbullet}
      node at (output)            [shift={( 0.007, -0.05 )}] {\Large \textbullet}
      node at (veloFeedbackNode)  [shift={( 0.007, -0.05 )}] {\Large \textbullet}
    ;

    %-- Summation signs --%
      \draw%--------------------------------------------------------------

      node at (sum1) [right = -6.6mm, below = .6mm] {$-$}
      node at (sum1) [right = -3mm, below = 3.9mm]  {$+$}
      node at (sum2) [right = -6.6mm, below = .6mm] {$+$}
      node at (sum2) [right = -3mm, below = 3.9mm]  {$-$}

    ;

  \end{tikzpicture}
		\centering
	\caption{Block diagram of the system}
	\label{blockDiagramSenseTool}
\end{figure}

The data is obtained through a test described in Appendix \ref{impulseResponseAppendix}, which gives the following result:

\begin{figure}[H]
	\centering
	\includegraphics[scale=0.6]{figures/PendRad}
	\caption{Angular position of the Cubli for the initial condition test}
	\label{cubliInitCondTest}
\end{figure}

As the operating angle goes from \si{-0,15\ to\ 0,15\ rad}, the behavior at this range will be better if the fit is done between this operating points.

The result of this fit is seen in \figref{SenseToolParameterEstimation}, where the final estimated parameters are \si{J_F=4,8 \cdot 10^{-3}\ kg \cdot m^2\ and\ B_F=7,7 \cdot 10^{-3}\ m \cdot s \cdot rad^{-1}}. The normed RMS error is \si{27,4\ \%} compared to the \si{31,7\ \%} shown in \figref{ParameterEstimationNewtonCubli}. The \si{4,3\ \%} error reduction is likely due to additional algorithms based on parameter sensitivity and frequency domain analysis, which are implemented in the Sense Tool toolbox.

\begin{figure}[H]
	\centering
	\includegraphics[scale=0.6]{figures/SenseToolParameterEstimation}
	\caption{Data from the test (red) and final fit with the new parameters (blue)}
	\label{SenseToolParameterEstimation}
\end{figure}

\subsection{Final Parameters}
The final parameters of the system can be seen in \ref{ParametersSystem}
\begin{table}[H]
	\begin{tabular}{|l|l|p{3cm}|}
		\hline %-----------------------------------------------------------------------------------
		\textbf{Parameter} &\textbf{Value} &\textbf{Units}\\
		\hline %-----------------------------------------------------------------------------------
		\si{m_w}         & \si{0,222}       &kg\\
		\hline
		%-----------------------------------------------------------------------------------
		\si{l_w}         & \si{0,096}       &m\\
		\hline %-----------------------------------------------------------------------------------
		\si{J_w}            & \si{0,601 \cdot 10^{-3}}	&\si{kg \cdot m^2}\\
		\hline  
		%-----------------------------------------------------------------------------------
		\si{B_w}         & \si{17,03 \cdot 10^{-6}}       &N \si{\cdot m \cdot s \cdot rad^{-1}}\\
		\hline
		%-----------------------------------------------------------------------------------
		\si{m_F}         & \si{0,548}       &kg\\
		\hline
		%-----------------------------------------------------------------------------------
		\si{l_F}         & \si{0,08498}       &m\\
		\hline %-----------------------------------------------------------------------------------
		\si{J_F}            & \si{4,8 \cdot 10^{-3}}	&\si{kg \cdot m^2}\\
		\hline %-----------------------------------------------------------------------------------
		\si{B_F}         & \si{7,7 \cdot 10^{-3}}       &N \si{\cdot m \cdot s \cdot rad^{-1}}\\
		\hline
	\end{tabular}
	\caption{Parameters of the whole system}
	\label{ParametersSystem}
\end{table}