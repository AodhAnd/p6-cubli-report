\subsection{The Optimization Problem}
The basic scheme for the optimization problem is given in \figref{SensToolSchema}.
%
\begin{figure}[H]
	\centering
	\includegraphics[scale=0.4]{figures/SensToolSchema}
	\caption{Schematic of the optimization problem}
	\label{SensToolSchema}
\end{figure}
%
The provided data is taken from an initial value test of the Cubli hanging down like a pendulum, see \appref{app:impulseResponseAppendix}.
As the operating angle goes from \si{-0,15\ to\ 0,15\ rad}, the behavior at this range will be better if the fit is done between this operating points. 

Furthermore, the nonlinear model is used to accurately describe the oscillatory behavior of the pendulum. The model is modified such that it describes the system as a regular pendulum without the dynamics of the reaction wheel, in order to match the test conditions under which the data was extracted, as seen in \figref{blockDiagramSenseTool}.
%
\begin{figure}[H]
	  \begin{tikzpicture}[ auto,
                       thick,                         %<--setting line style
                       node distance=1.5cm,             %<--setting default node distance
                       scale=0.75,                     %<--|these two scale the whole thing
                       every node/.style={scale=0.62}, %<  |(always change both)
                       >/.tip={Triangle[angle=40:5pt]}
                       ]

    %-- Blocks creation --%
    \draw
      % DIRECT TERM %
      node[shape=coordinate][](input1) at (0,0){}
      node[shape=coordinate][](feedForward) at (0.5,0){}
      node(sum1) at (2.25,0) [sum]{\si{\sum}}
      node(sum2) at (3.75,0) [sum]{\si{\sum}}

      node(torque2rotacc1) at (6.85,0) [block]{\large \si{\frac{1}{J_F + J_w + m_w \cdot {l_w}^{2}}}}

      node(integration1) at (10.75,0) [block] {\large \si{\frac{1}{s}}}
      node(integration2) at (14.2,0) [block] {\large \si{\frac{1}{s}}}

      node[shape=coordinate][](output) at (15.3,0){}
      node[shape=coordinate][](veloFeedbackNode) at (11.8,0){}
      
    ;
    \draw
      % FEEDBACKS %
      node(veloFeedback) at (7,-1) [block] {\large \si{B_F}}
      node(angleFeedback) at (8,-2) [block] {\large \si{(m_F \cdot l_F + m_w \cdot l_w)g}}
    ;
    %-- Block linking --%
    % INPUT %
    \draw[-](input1)        -- node{\large \si{\tau_m(s)}}(feedForward);
    \draw[->](feedForward)  -- (sum1);

    % OUTPUT %
    \draw[-](integration2)  -- (output);
    \draw[->](output)       -- node {\large \si{\theta_{F}(s)}} (17,0);

    % DIRECT TERM %
    \draw[->] (sum1)            -- (sum2);
    \draw[->] (sum2)            -- (torque2rotacc1);
    \draw[->] (torque2rotacc1)  -- node{\large \si{\alpha_F(s)}}(integration1);
    \draw[->] (integration1)    -- node{\large \si{\omega_F(s)}}(integration2);

    % FEEDBACKS

    \draw[->] (output)           |- (angleFeedback);
    \draw[->] (angleFeedback)    -| (sum1);

    \draw[->] (veloFeedbackNode) |- (veloFeedback);
    \draw[->] (veloFeedback)     -| (sum2);

    %-- Nodes --%
    \draw%--------------------------------------------------------------
      node at (input1)            [shift={(-0.04, -0.05 )}] {\Large \textopenbullet}
      node at (output)            [shift={( 0.007, -0.05 )}] {\Large \textbullet}
      node at (veloFeedbackNode)  [shift={( 0.007, -0.05 )}] {\Large \textbullet}
    ;

    %-- Summation signs --%
      \draw%--------------------------------------------------------------

      node at (sum1) [right = -6.6mm, below = .6mm] {$-$}
      node at (sum1) [right = -3mm, below = 3.9mm]  {$+$}
      node at (sum2) [right = -6.6mm, below = .6mm] {$+$}
      node at (sum2) [right = -3mm, below = 3.9mm]  {$-$}

    ;

  \end{tikzpicture}
	\centering
	\caption{Block diagram of the system as a regular pendulum with the wheel fixed to the frame.}
	\label{blockDiagramSenseTool}
\end{figure}
%
 In order to minimize the difference between the data points measured in test and the output of the model, a function to describe such a relationship is needed. The performance function used to describe goodness of the fit, is a mean square error function.
%
\begin{flalign}
	\eq{P(\vec{\theta})} {\frac{1}{2N}\sum_{k = 1}^{N} \left(\vec{y}(kT) - \vec{y_m}(kT, \vec{\theta})\right)^2 } &
\label{performanceFunction}
\end{flalign}
%
\hspace{6mm} Where:\\
\begin{tabular}{ p{1cm} l l l}
& \si{\vec{\theta}}   & is the parameter(s) to be adjusted                  & \\
& \si{N}              & is the degrees of freedom for each parameter        & \\
& \si{k}              & is the sample indexes, \si{k=1,\ 2,} ...\si{,\ N}   & \\
& \si{T}              & is the sampling time                                & \\
& \si{\vec{y}}        & is the test measurement output vector               & \\
& \si{\vec{y_m}}      & is the model output vector                          & \\
\end{tabular}

A normal mean square error function is only divided by the degrees of freedom, \si{N}, but in tis case is divided by \si{2N} to cancel out the factor which arises when computing its gradient. This only gives the function a constant offset and does not have any impact when minimizing it.