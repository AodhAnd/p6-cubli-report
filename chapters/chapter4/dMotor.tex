\section{Motor}
\label{sec:Motor}


\subsection{Brushless DC-Motor}
To control the reaction wheel a brushless DC-motor is used. \fxnote{put datasheet on the cd appendix}
The brushless motor used on the Cubli setup is a EC 45 flat 50 Watt (part no. 251601). 
%can take a nominal current of 2.33 A, and 23 A at very short peaks.It has a mechanical time constant of 12,4 ms \cite{MaxonMotors}. The nominal current puts a limit on the control signal that can be sent to the motor control board.\fxnote{Should this control signal be mentioned here?}.
%\fxnote{Should there be made a table listing the motors data instead of writing a paraghap with the info?}


\begin{table}[H]
	\centering
	\begin{tabular}{|p{5cm}|p{2.3cm}|}
		\hline%------------------------------------------------------------------------------------
		\textbf{Motor Data}                        &  \textbf{Value} \unitWh{Unit}  \\
		\hline%------------------------------------------------------------------------------------
%		Max voltage                               &  24 \unitWh{V}  	\\
%		\hline%------------------------------------------------------------------------------------
		Nominal current                   		  &  2,33 \unitWh{A}	\\
		\hline%------------------------------------------------------------------------------------
		Motor constant (\si{K_t})				 &  33,5 \unitWh{\frac{\si{m\cdot Nm}}{\si{A}}}  \\
		\hline%------------------------------------------------------------------------------------
		Mechanical time contant                 &  12,4 \unitWh{ms}  \\
		\hline%------------------------------------------------------------------------------------
	\end{tabular}
	\caption{Important parameters of the brushless DC-motor}
	\label{BrushlessDCMotorTable}
\end{table}


\subsection{Motor Control Board}
Connected between the BeagleBone and the Brushless DC-motor is a ESCON Module 50/5, which is a motor control board. It is specifically made to work with ESCON motors (which the EC 45 flat is) and can be configured with a program provided by Maxon called ESCON studio.\cite{ESCONStudio}
The control board has a setup selected by the previous group working with the Cubli setup.\\ 
%The PWM signal has a range of 10 \% to 90 \%. 
Inside the board is a 4-quadrant configuration and is controlled by a PWM signal which is limited to the range 10 \% to 90 \%. Through the software the current is limited to 4 A at 90 \% and -4 A at 10 \%.\\
The control board is set to run with a current control loop, which will keep the set current while the motor runs. It takes a signal, representing a current, as an input and then regulates the voltage over the motor to counteract back-EMF.\fxnote{double check the emf thing, i might be wrong, B}

\begin{table}[H]
	\centering
	\begin{tabular}{|p{7cm}|p{2.3cm}|}
		\hline%------------------------------------------------------------------------------------
		\textbf{ESCON Module 50/5}                &  \textbf{Value} \unitWh{Unit}  \\
		\hline%------------------------------------------------------------------------------------
		Nominal output current                    &  5 \unitWh{A}  	\\
		\hline%------------------------------------------------------------------------------------
		Peak current (<20 s)                    &  15 \unitWh{A}	\\
		\hline%------------------------------------------------------------------------------------
		PWM frequency 							 &  53,6 \unitWh{kHz}  \\
		\hline%------------------------------------------------------------------------------------
		Sample Rate of PI current controller      &  53,6 \unitWh{kHz}  \\
		\hline%------------------------------------------------------------------------------------
	\end{tabular}
	\caption{Important parameters of the motor control board}
	\label{MotorControlBoardTable}
\end{table}

\subsection{Jump Brake}
The last important part included on the system is the braking mechanism, which is used for making the frame go from rest to vertical position. 

To perform this task the motor spins up the reaction wheel and when it has enough kinetic energy the braking mechanism suddenly brake it. This makes that all the energy of the wheel is then transfer to the frame, which reaches the vertical position.
