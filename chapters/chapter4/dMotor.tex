\section{Actuators}\label{sec:Motor}
There are two actuators in the system, the brushless DC-motor as the main actuator which controls the stability and the servo motor which brakes on the wheel to raise the frame.

\subsection{Brushless DC-Motor}
To control the reaction wheel a brushless DC-motor is used. \fxnote{put datasheet on the cd appendix}
The brushless motor on the Cubli setup is an EC 45 flat 50 Watt (part no. 251601). Additionally the motor features three Hall-sensors for angular velocity measurements.
%can take a nominal current of 2.33 A, and 23 A at very short peaks.It has a mechanical time constant of 12,4 ms \cite{MaxonMotors}. The nominal current puts a limit on the control signal that can be sent to the motor control board.\fxnote{Should this control signal be mentioned here?}.
%\fxnote{Should there be made a table listing the motors data instead of writing a paraghap with the info?}


\begin{table}[H]
	\centering
	\begin{tabular}{|p{4.8cm}|p{3.3cm}|}
		\hline%------------------------------------------------------------------------------------
		\textbf{Motor Data}                        &  \textbf{Value} \unitWh{Unit}  \\
		\hline%------------------------------------------------------------------------------------
%		Max voltage                               &  24 \unitWh{V}  	\\
%		\hline%------------------------------------------------------------------------------------
		Nominal current                   		  &  2,33 \unitWh{A}	\\
		\hline%------------------------------------------------------------------------------------
		Motor constant (\si{K_t})				 &  33,5 \unitWh{N\cdot m \cdot A^{-1}}  \\
		\hline%------------------------------------------------------------------------------------
		Mechanical time constant                 &  12,4 \unitWh{ms}  \\
		\hline%------------------------------------------------------------------------------------
	\end{tabular}
	\caption{Important parameters of the brushless DC-motor}
	\label{BrushlessDCMotorTable}
\end{table}


\subsection{Motor Control Board}
There is a Motor Control Board, Maxon ESCON Module 50/5, connected between the BeagleBone and the Brushless DC-Motor. It is specifically made to work with ESCON motors and can be configured with a program provided by Maxon called ESCON studio.\cite{ESCONStudio}\\ 
%The PWM signal has a range of 10 \% to 90 \%. 
Inside the board is a 4-quadrant configuration.\\
The control board is set to run with a current control loop, which will ensure that the set current is reached while the motor runs\fxnote{Reference to material providing proof that the current loop is fast enough.}.\\
The reference current is sent as a PWM signal with its duty cycle configured within the range of 10 \% to 90 \% corresponding to \SI{4}{A} at 90 \% and \SI{-4}{A} at 10 \% see \appref{MaxonControlESCON}\fxnote{make sure that the appref links appropriately to the information in the appendix}.

\begin{table}[H]
	\centering
	\begin{tabular}{|p{7cm}|p{2.3cm}|}
		\hline%------------------------------------------------------------------------------------
		\textbf{Characteristics}                 &  \textbf{Value} \unitWh{Unit}  \\
		\hline%------------------------------------------------------------------------------------
		Nominal output current                   &  5 \unitWh{A}  	\\
		\hline%------------------------------------------------------------------------------------
		Peak current (<20 s)                     &  15 \unitWh{A}	\\
		\hline%------------------------------------------------------------------------------------
		Current control PWM frequency 				   &  53,6 \unitWh{kHz}  \\
		\hline%------------------------------------------------------------------------------------
		Sample Rate of PI current controller     &  53,6 \unitWh{kHz}  \\
		\hline%------------------------------------------------------------------------------------
	\end{tabular}
	\caption{Important parameters of the motor control board}
	\label{MotorControlBoardTable}
\end{table}

\subsection{Braking System}
Also included in the system is the braking mechanism, which is used for making the frame go from a resting position to vertical position.

To perform this task the motor spins up the reaction wheel and when it has enough kinetic energy the braking mechanism suddenly brake it. The inertia of the wheel is thus transfered to the frame, in order to raise it to standing position.

The actuator used to brake the reaction wheel Cubli setup is a Hitec HS225 Mighty Mini Servomotor\fxnote{Add reference(did you mean citation?)}. The brake system is however not used in this project, since the jump up procedure is out of its scope.

%\fxnote{For gods sake fix this table!..:p}
%\begin{table}[H]
%	\centering
%	\begin{tabular}{|l|llll|}
%		\hline%------------------------------------------------------------------------------------
%		\textbf{Characteristics}           & \textbf{Values} &                      &  &\\
%		\hline%------------------------------------------------------------------------------------
%		Reaction time                      & @\SI{4,8}{V}    &\si{0,14/60} \si{s\cdot deg^{-1}} &and @\SI{6,0}{V} \si{0,11/60} &\si{s\cdot deg^{-1}} \\
%		\hline%------------------------------------------------------------------------------------
%		Speed                              & @\SI{4,8}{V}    &\SI{7,48}{rad\cdot s^{-1}} &and @\SI{6,0}{V} \si{9,52} &\si{rad\cdot s^{-1}}             \\
%		\hline%------------------------------------------------------------------------------------
%		Torque                             & @\SI{4,8}{V}    &\SI{3,9}{kgF\cdot cm} &and @\SI{6,0}{V} \si{4,8} &\si{kgF\cdot cm} \\
%		\hline%------------------------------------------------------------------------------------
%		Size                							 &  \multicolumn{4}{l|}{\si{32,26\ x\ 16,76\ x\ 31,00} \si{mm}}             \\
%		\hline%------------------------------------------------------------------------------------
%		Weight                             &  \SI{27,94}{g}  &                       &     &        \\
%		\hline%------------------------------------------------------------------------------------
%	\end{tabular}
%	\caption{Hitec HS225 Mighty Mini Servomotor Specifications}
%	\label{HitecHS225Servomotor}
%\end{table}