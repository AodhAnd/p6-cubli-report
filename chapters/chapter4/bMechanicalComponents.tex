\section{Mechanical Components}
The mechanical part of the whole Cubli is composed of two important elements, which are the main components of a reaction wheel inverted pendulum: the frame and the reaction wheel.

\subsection{Frame}
The frame is made of aluminum with dimensions 17x16x0,5 \si{cm}  with two cross connections between opposing corners. The frame mounting is centered around the middle of the frame.
To keep the weight down on the frame, a large area of the frame is milled out. That composes the body of the Cubli. 

As it can only be balanced in one direction, it is attached to the base on one of its vertices to avoid it from falling in any of the other directions.

%It plays an important role since it acts like the inverted pendulum of the system and, moreover, all the electronics and the motor needed are attached to it.\fxnote{Isn't the motor rather attached to the wheel (both in reality and in the model) ? J.}
%This last aspect has to be taken into account when finding the model, as its moment of inertia, center of gravity and mass change.

\subsection{Reaction Wheel}
%A reaction wheel is a device used in many applications, such as attitude control in spacecrafts, when a change in rotation or keeping a certain attitude is needed.\fxnote{source?}
%The use of this kind of wheels allows these small variations in the rocket orientation, and thus reducing fuel needed to do this task.\fxnote{Do we really want a comparison with rockets or satellites?}
In the particular case of the Cubli, the reaction wheel is made of brass with most of the mass in a ring at its outer edge and two cross connections through its center.

The reaction wheel is coupled to the axis of a motor, through its center of rotation. When the wheel turns, its change of velocity creates a torque on the system that is transmitted with opposite direction to the body due to the conservation of angular momentum.