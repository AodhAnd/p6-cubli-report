\section{Mechanical Components}
The mechanical section of the whole Cubli is composed by two important elements, which are the main components of a reaction wheel inverted pendulum: the frame and the reaction wheel.



\subsection{Frame}



The frame is in aluminum with the dimension of 17x16x0,5cm  with two cross connections between opposing corners. The frame mounting is centered around the middle of the frame.
To keep the weight down on the frame, a large area of the frame been milled out. That composes the body of the Cubli. 

As it can only be balanced in one direction, it is attached to the base on one of its vertexes to avoid it from falling in any of the other directions.

It plays an important role since it acts like the inverted pendulum of the system and, moreover, all the electronics and the motor needed are attached to it. This last aspect has to be taken into account when finding the model, as its moment of inertia, center of gravity and mass change.

\subsection{Reaction Wheel}
A reaction wheel is a device used in many applications, such as attitude control in spacecrafts, specially when a change in rotation is needed. The use of this kind of wheels allows these small variations in the rocket orientation, and thus reducing fuel needed to do this task.\fxnote{Do we really want a comparison with rockets or satellites?}

 This behavior can be achieved since the wheel is coupled to the axis of a motor, through its center of rotation. When the wheel turns, it creates a torque on the system that is transmitted with opposite direction to the body due to the conservation of angular momentum.

In the particular case of Cubli, the reaction wheel is made of brass with most of the mass in a ring at its outer edge and two cross connections through its center.
