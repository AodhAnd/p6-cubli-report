\chapter{System Description}\label{systemDescription}
%One-frame Cubli is a non-linear unstable system which is capable of standing on one of its corners as long as it is properly controlled, using the inertia created in the reaction wheel present in the setup.
The system is composed of a frame and reaction wheel as the main objects to be controlled.\\ The actuation is done with a brushless dc motor which actuates both on the frame and the wheel.\\ A board with microcontroller is mounted on the frame through a connecting and breakout board on which a dedicated motor controller board is also mounted\fxnote{source Simon connector board}.\\ The frame is fixed to a base-plate, connected with a potentiometer provided for direct angle measurements.\\ Two additional sensor breakout boards, including an integrated circuit with gyro and accelerometer, are located at the top of the frame. These are to be used for angle and velocity measurements independently of the angle with respect to the base-plate.\\ A servomotor is also attached controlling the brake system, which consists of an arm capable of blocking the wheel by hitting one of the two break-blocks mounted on the edge of the reaction wheel.

\fxnote{insert picture of the system with its components, featuring numbers corresponding to table 3.1}

%\begin{figure}[H]
%  \centering
%	\includegraphics[scale=.8]{figures/SystemComponents}
%	\caption{}
%	\label{fig:SystemComponents}
%\end{figure}

This system is composed of several hardware parts, listed in the following table. \fxnote{reconstruct sentence to match both figure and table  .. Maybe put the table on the picture or make it smaller.. maybe by the side of the picture.. do something - watch the caption placement}

\begin{table}[H]
	\begin{tabular}{|l|p{6.7cm}|}
		\hline %-----------------------------------------------------------------------------------
		\textbf{No.} &\textbf{Part} 			\\
		\hline %-----------------------------------------------------------------------------------
		1            & Frame           			\\
		\hline %-----------------------------------------------------------------------------------
		2            & Reaction wheel      		\\
		\hline %-----------------------------------------------------------------------------------
		3            & Microcontroller (BeagleBone Black)  \\
		\hline %-----------------------------------------------------------------------------------
		4            & Potentiometer			\\
		\hline %-----------------------------------------------------------------------------------
		5            & Sensor breakout boards (Inertia Measurement Units (IMU))       			\\
		\hline %-----------------------------------------------------------------------------------
		6            & Brushless DC-motor   	\\
		\hline %-----------------------------------------------------------------------------------
		7            & Motor control board     	\\
		\hline %-----------------------------------------------------------------------------------
		8            & Servo brake system 		    	\\
		\hline %-----------------------------------------------------------------------------------
		9            & Connecting and breakout board		    	\\
		\hline %-----------------------------------------------------------------------------------
	\end{tabular}0
	\caption{Main parts of the Cubli setup}
\label{TableAAUCubliComponent}
\end{table}

%The BeagleBone Black, the break servomotor, the Maxon motor control board and the motor itself are mounted directly on the frame. On the motor shaft a freewheel is mounted. 

%There are 2 pieces of IMU mounted on the frame to measure the acceleration and angular velocity of that frame. The latter is fixed to the platform via a potentiometer so the angular position of the frame can be measured.

%To connect the BeagleBone Black to the other units like the Maxon motor control board and the IMU and potentiometer there is a connecting and breakout board.\fxnote{Shouldn't we state who has made it ? J.}

%The Cubli at Aalborg University is a complete finished mechanical and electronic system and has been built by others.
%From the previous groups that have been working on it, some of the parameters have been found.
%To confirm and investigate the configuration of the mechanical parts changes has been made to the Cubli and the new parameters have not been measured and verified.

