\chapter{IMU Setup Retrieval} \label{app:IMUSetupRetrieval}
\textbf{Name: Group 630}\\
\textbf{Date: 30/04 - 2016}

\subsubsection{Purpose}
Determining how the IMU is configured on the Cubli to help analyzing the data that is measured from it.

\subsubsection{Setup}
The Cubli is plugged through USB to a PC which has the ability to connect via Secure SHell (SSH) to the Beaglebone Black.\\
A program is written with parts of the given code to communicate with the IMU from the Beaglebone Black and retrieve the registers data from the sensor chip.
                                                     
\subsubsection{List of Equipment}
\begin{table}[H]
	\centering
\begin{tabular}{|p{10cm}|p{4cm}|}
\hline%------------------------------------------------------------------------------------
  \textbf{Instrument}                &  \textbf{Type} \\
\hline%------------------------------------------------------------------------------------
  Computer                           &  Asus UX301LA  \\
\hline%------------------------------------------------------------------------------------
\end{tabular}
\end{table}

\subsubsection{Procedure}

\begin{enumerate}
  \item Plug the Beaglebone Black through USB and wait until the blue LEDs on the Beaglebone Black start blinking slowly.
  \item Send the binary compiled program to the board.
  \item Connect to a distant terminal on the Beaglebone Black through SSH and launch the program.
  \item Read the output of the program.
\end{enumerate}
\subsubsection{Results}
From the output of the program, the following table is built.

\begin{table}[H]
	\centering
\begin{tabular}{|l|l|l|}
\hline%--------------------------------------------------------------------------------------------------
  \textbf{Register}       &  \textbf{Hex Value on IMU 1}  & \textbf{Hex Value on IMU 2} \\
\hline%--------------------------------------------------------------------------------------------------
  CONFIG                  &   0x00                                & 0x00 \\
\hline%--------------------------------------------------------------------------------------------------
  GYRO\_CONFIG            &   0x00                                & 0x00 \\
\hline%--------------------------------------------------------------------------------------------------
  ACCEL\_CONFIG           &   0x00                                & 0x00 \\
\hline%--------------------------------------------------------------------------------------------------
\end{tabular}
\end{table}

Having a 0 in the CONFIG registers means that both IMU use a low pass filter for the noise on their output with the highest possible frequency of \SI{260}{Hz} for the accelerometers and \SI{256}{Hz} and that the accelerometers have no delay while the gyroscopes have a rather small delay of \SI{0,98}{ms}.

The GYRO\_CONFIG and ACCEL\_CONFIG three most significant bits indicate that no self-test are asked to the sensor. The two next give information on the pre-selected range of measurement of these sensors.\\
Since there are all set to 0, gyroscopes will measure data within \si{\pm\ }\SI{250}{^\circ \cdot s^{-1}}, whereas accelerometers will present values in the range of \si{\pm\ 2g} (g being the Earth's gravitational acceleration).\\
Eventually, the three least significant bits in ACCEL\_CONFIG being set to 0, the digital filter on the accelerometers are disabled and the output value is put to 0 after each sample.