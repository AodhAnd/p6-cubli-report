\chapter{Pendulum Behavior Test}\label{app:impulseResponseAppendix} 

\textbf{Name: Group 630}\\
\textbf{Date: 16/03 - 2016}

\subsubsection{Purpose}
Observing the behavior of the Cubli when hanging upside down. Data is then used to estimate some of the parameters of the Cubli.

\subsubsection{Setup}
The Cubli is put upside down under a table with 2 clamps placed at each side of the bottom plate of the Cubli setup.
The wheel is being held in a fixed position with a strip tied to it and the frame. The probe chosen is a 1:1 and is connected to the potentiometer with probe to yellow cable and ground clamp to brown cable. The power supply has to be connected, and turned on, to the Cubli in order to get readouts from the potentiometer.
\begin{figure}[H] 
	\centering 
	\includegraphics[scale=0.2]{figures/impulseResponseSetup2}
	\caption{The Cubli setup hanging upside down beneath a table during the impulse response test}
	\label{impulseResponseTestPicture}
\end{figure} 

\subsubsection{List of Equipment}
\begin{table}[H]
	\begin{tabular}{|l|l|p{4.3cm}|}
		\hline%------------------------------------------------------------------------------------
		\textbf{Instrument}                        &  \textbf{AAU-no.}  &  \textbf{Type}       \\
		\hline%------------------------------------------------------------------------------------
		Oscilloscope                              &  61604             &  Agilent 54621A		  \\
		\hline%------------------------------------------------------------------------------------
		Dedicated Power Supply of Cubli \small{(24 V - 3 A)} &               &  XP Power, AEB70US24 \\
		\hline%------------------------------------------------------------------------------------
		Probe               &             		&          1:1  \\
		\hline%------------------------------------------------------------------------------------
		2x Clamp                &  			            &          							   \\
		\hline%------------------------------------------------------------------------------------
	\end{tabular}
\end{table}

\subsubsection{Procedure}
\begin{enumerate}
	%\item Turn on the power supply
	\item Place the setup upside-down and place the frame touching the base.
	\item Let the Cubli fall and swing until it stops.
	\item Use the oscilloscope to measure the changes in the potentiometer.
	\item Collect all the data and plot it in Matlab.	
\end{enumerate}

\subsubsection{Results}
The results of the experiment can be seen in \figref{PendVolt} and \figref{PendRad}. The second graph is obtained using the conclusions from \appref{app:potentiometerRes} explained in Section \ref{sec:Sensors}.

\begin{minipage}{\linewidth}
	\begin{minipage}{0.45\linewidth}
		\begin{figure}[H]
			\includegraphics[scale=.53]{figures/PendVolt}
			\centering
			\vspace{-.4cm}
			\captionsetup{justification=centering}
			\captionof{figure}{Raw data taken from the potentiometer}
			\label{PendVolt}
		\end{figure}%\vspace{-5mm}
	\end{minipage}
	\hspace{0.03\linewidth}
	\begin{minipage}{0.45\linewidth}
		\begin{figure}[H]\vspace{-4mm}
			\includegraphics[scale=.53]{figures/PendRad}
			\centering
			\vspace{-.4cm}
			\captionsetup{justification=centering}
			\captionof{figure}{Angular position of the frame}
			\label{PendRad}
		\end{figure}%\vspace{-5mm}
	\end{minipage}
\end{minipage}

\subsubsection{Note}
During this experiment it was observed that if the frame was released from the left position (the right upper side on \figref{impulseResponseTestPicture} since the Cubli is upside down), the frame would hit the rubber pad on the other side. This behavior was not observed when releasing the Cubli from the right position (left upper corner).

