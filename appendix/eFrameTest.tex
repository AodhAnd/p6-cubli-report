\chapter{Frames: Mass and position of Center of Mass }\label{MassFrameCenterOfMass} 
\textbf{Name: Group 630}\\
\textbf{Date: 15/03 - 2016}

\subsubsection{Purpose}
Measuring Mass of the Frame and Center of Mass of the Frame.

\subsubsection{Setup}
\begin{figure}[H]
  \centering
	\includegraphics[scale=1]{figures/LabSetupLinearityTest.pdf}
	\caption{Setup diagram}
	\label{LabSetupRangeTest}
\end{figure}\vspace{-5mm}
%
%\subsubsection{List of Equipment}
%\begin{table}[H]
%	\begin{tabular}{|l|l|p{4.3cm}|}
%		\hline%------------------------------------------------------------------------------------------------------------
%		\textbf{Instrument}                                  &  \textbf{AAU-no.}  &  \textbf{Type}                       \\
%		\hline%------------------------------------------------------------------------------------------------------------
%		Multimeter                                           &  TBD           &  Fluke		                   \\
%		\hline%------------------------------------------------------------------------------------------------------------
%		Dedicated Power Supply of Cubli \small{(24 V - 3 A)} &                    &  XP Power, AEB70US24                 \\
%		\hline%------------------------------------------------------------------------------------------------------------
%		Digital Protractor                                   &  None               &  TBD\fxnote{find the probe used}     \\
%		\hline%------------------------------------------------------------------------------------------------------------
%	\end{tabular}
%\end{table}
%
%\subsubsection{Procedure}
%\begin{enumerate}
%  \item Make the setup with connections as seen on \figref{LabSetupRangeTest}, with minus on brown and plus on yellow of the potentiometer.
%  \item Setting the multimeter to measure DC mV.
%  \item Balance the frame in upright equilibrium position measuring the angle and voltage.
%  \item measuring every 10 degree the voltage of the potentiometer around Equilibrium point and also min and max angle voltage.
%\end{enumerate}
%
%
%\subsubsection{Results}
%\begin{table}[H]
%	\begin{tabular}{|l|l|p{4.3cm}|}
%		\hline%------------------------------------------------------------------------------------------------------------
%		\textbf{Angle form equilibrium deregee}       &  \textbf{mV}         \\
%		\hline%------------------------------------------------------------------------------------------------------------
%		-41,7                                         & 0,066               \\
%		\hline%------------------------------------------------------------------------------------------------------------
%		-40,0 										  & 0,067               \\
%		\hline%------------------------------------------------------------------------------------------------------------
%		-38,5                              			  & 4,25               \\
%		\hline%------------------------------------------------------------------------------------------------------------
%		-30,0                              			  & 50,55               \\
%		\hline%------------------------------------------------------------------------------------------------------------
%		-20,0                                         & 103,66               \\
%		\hline%------------------------------------------------------------------------------------------------------------
%		-10,0 										  & 159,20               \\
%		\hline%------------------------------------------------------------------------------------------------------------
%		0                               			  & 213,64               \\
%		\hline%------------------------------------------------------------------------------------------------------------
%		10,0                                          & 264,66               \\
%		\hline%------------------------------------------------------------------------------------------------------------
%		20,0 										  & 317,95               \\
%		\hline%------------------------------------------------------------------------------------------------------------
%		30,0                              			  & 370,74               \\
%		\hline%------------------------------------------------------------------------------------------------------------
%		40,0                                          & 425,10               \\
%		\hline%------------------------------------------------------------------------------------------------------------
%		48,65 										  & 472,11               \\
%		\hline%------------------------------------------------------------------------------------------------------------		
%	\end{tabular}
%\end{table}
%	
%			
%\subsubsection{Results in a graf}
%\begin{figure}[H] 
%	\centering 
%	\includegraphics[scale=0.7]{figures/linearityOfPotmeterTest2-1}
%	\caption{Raw test data plot}
%	\label{linearityOfPotmeterTest2-1}
%\end{figure}
%
