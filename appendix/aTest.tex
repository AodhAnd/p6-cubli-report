\chapter{Comparison Between Linear Model and Real System}\label{comparisonLinModelReal} 
\textbf{Name: Group 630}\\
\textbf{Date: 07/03 - 2016}

\subsubsection{Purpose}
Compare the response of the real setup with the simulation given by the theoretical linearized model when it falls from the vertical position, \si{theta_F=0}.

\subsubsection{Procedure}
\begin{enumerate}
	\item Turn on the power supply
	\item Keep the Cubli near the vertical position
	\item Let the Cubli fall
	\item Use the oscilloscope to measure the changes in the potentiometer 
	\item Get the measurements and plot them in Matlab
	\item Plot the result of the simulations in the same figure and compare them

\end{enumerate}

\subsubsection{Results}
\begin{figure}[H] 
	\centering 
	\includegraphics[scale=0.9]{figures/comparisonRealModel}
	\caption{Comparison between the real behavior and the simulation of the linearized model}
	\label{comparisonRealModel}
\end{figure} 

The result of the experiment (\figref{comparisonRealModel}) shows that the response of the real system has several differences with the one from the simulation.

The fist one is the presence of oscillations in the real response curve. This behavior is due to a small bounce that the frame does when it reaches the base.

Another one is the final position of the frame, but it is due to the existence of a piece of foam at this position in the real case (to avoid the Cubli to hit the base).

The other main difference is the shape of the curve, since the simulation is slower than the real case.

\subsubsection{Conclusions}

