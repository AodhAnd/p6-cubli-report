%Creates the aau titlepage
\newcommand{\aautitlepage}[3]{%
  {
    %set up various length
    \ifx\titlepageleftcolumnwidth\undefined
      \newlength{\titlepageleftcolumnwidth}
      \newlength{\titlepagerightcolumnwidth}
    \fi
    \setlength{\titlepageleftcolumnwidth}{0.5\textwidth-\tabcolsep}
    \setlength{\titlepagerightcolumnwidth}{\textwidth-2\tabcolsep-\titlepageleftcolumnwidth}
    %create title page
    \thispagestyle{empty}
    \noindent%
    \begin{tabular}{@{}ll@{}}
      \parbox{\titlepageleftcolumnwidth}{
        \iflanguage{danish}{%
          \includegraphics[width=\titlepageleftcolumnwidth]{setup/aau_logo_da.pdf}
        }{%
          \includegraphics[width=\titlepageleftcolumnwidth]{setup/aau_logo_en.pdf}
        }
      } &
      \parbox{\titlepagerightcolumnwidth}{\raggedleft\sf\small
        #2
      }\bigskip\\
       #1 &
      \parbox[t]{\titlepagerightcolumnwidth}{%
      \textbf{Abstract:}\smallskip\par
        \fbox{\parbox{\titlepagerightcolumnwidth-2\fboxsep-2\fboxrule}{%
          #3
        }}
      }\\
    \end{tabular}
    \vfill
    \vspace{-0.5cm}
    \iflanguage{danish}{%
      \noindent{\footnotesize\emph{Rapportens indhold er frit tilgængeligt, men offentliggørelse (med kildeangivelse) må kun ske efter aftale med forfatterne.}}
    }{%
      \noindent{\footnotesize\emph{The content of this report is freely available, but publication (with reference) may only be pursued due to agreement with the author.}}
    }
    \clearpage
  }
}

%Create english project info
\newcommand{\englishprojectinfo}[8]{%
  \parbox[t]{\titlepageleftcolumnwidth}{
    \textbf{Title:}\\ #1\bigskip\par
    \textbf{Theme:}\\ #2\bigskip\par
    \textbf{Project Period:}\\ #3\bigskip\par
    \textbf{Project Group:}\\ #4\bigskip\par
    \textbf{Participant(s):}\\ #5\bigskip\par
    \textbf{Supervisor(s):}\\ #6\bigskip\par
    \textbf{Copies:} #7\bigskip\par
    \textbf{Page Numbers:} Fucking mange!\bigskip\par
    \textbf{Date of Completion:}\\ #8
  }
}

%Create danish project info
\newcommand{\danishprojectinfo}[8]{%
  \parbox[t]{\titlepageleftcolumnwidth}{
    \textbf{Title:}\\ #1\bigskip\par
    \textbf{Theme:}\\ #2\bigskip\par
    \textbf{Project Period:}\\ #3\bigskip\par
    \textbf{Project Group:}\\ #4\bigskip\par
    \textbf{Participants:}\\ #5\bigskip\par
    \textbf{Supervisor:}\\ #6\bigskip\par
    \textbf{Copies:} #7\bigskip\par
    \textbf{Page Numbers:} ??
    \bigskip\par
    \textbf{Date of Completion:}\\ #8
  }
}


\newcommand{\iic}[0]{I²C }

%%%%%%%%%%%%%%%%%%%%%%%%%%%%%%%%%%%%%%%%%%%%%%%%
%            An example environment            %
%%%%%%%%%%%%%%%%%%%%%%%%%%%%%%%%%%%%%%%%%%%%%%%%
\theoremheaderfont{\normalfont\bfseries}
\theorembodyfont{\normalfont}
\theoremstyle{break}
\def\theoremframecommand{{\color{aaublue!50}\vrule width 5pt \hspace{5pt}}}
\newshadedtheorem{exa}{Example}[chapter]
\newenvironment{example}[1]{%
		\begin{exa}[#1]
}{%
		\end{exa}
}

\makeatletter
\newcommand{\ChapterOutsidePart}{%
   \def\toclevel@chapter{-1}\def\toclevel@section{0}\def\toclevel@subsection{1}}
\newcommand{\ChapterInsidePart}{%
   \def\toclevel@chapter{0}\def\toclevel@section{1}\def\toclevel@subsection{2}}
\makeatother

\usepackage{bookmark}

\usepackage{mathtools}
\DeclarePairedDelimiter{\ceil}{\lceil}{\rceil}


%%%%%%%%%%%%%%%%%%%%%%%%%%%%%%%%%%%%%%%%%%%%%%%%%%%%%
%                  USEFULL MACROES                  %
%%%%%%%%%%%%%%%%%%%%%%%%%%%%%%%%%%%%%%%%%%%%%%%%%%%%%
%Units:
\newcommand{\unit}[1]{&& \left[\si{#1}\right]} %\newcommand{\unit}[1]{[\si{#1}]}             <<| Use these if you want equations to be
\newcommand{\unitWh}[1]{[\si{#1}]}             %\newcommand{\eq}[2]{&&\si{#1} &= \si{#2}&&}  <<| centered.. .. will be appear scrambled
\newcommand{\numUnit}[1]{\ \si{#1}&}           %                                               | from one equation to the next though..
%Equation:                                     %                                               | and does not work with long equations.. :/
\newcommand{\eq}[2]{\si{#1} &= \si{#2}}
\newcommand{\arw}{&& &\Updownarrow&&}
%Text:
\newcommand{\tx}[1]{\text{#1}}


%%%%%%%%%%%%%%%%%%%%%%%%%%%%%%%%%%%%%%%%%%%%%%%%%%%%%
%                  REFERENCES                       %
%%%%%%%%%%%%%%%%%%%%%%%%%%%%%%%%%%%%%%%%%%%%%%%%%%%%%

%Chapter
\newcommand{\chapref}[1]{\emph{Chapter \ref{#1}}} %:\textbf{ \nameref{#1}}}}
%Section
\newcommand{\secref}[1]{\emph{Section \ref{#1}}} %:\textbf{ \nameref{#1}}}}
%subSection
\newcommand{\subsecref}[1]{\emph{Subsection \ref{#1}}} %:\textbf{ \nameref{#1}}}}
%Appendix
\newcommand{\appref}[1]{\emph{Appendix \ref{#1}}} %:\textbf{ \nameref{#1}}}}
%Listings
\newcommand{\coderef}[1]{\emph{Listings: \ref{#1}}}
%Figure:
\newcommand{\figref}[1]{\emph{Figure \ref{#1}}}
%Table:
\newcommand{\tableref}[1]{\emph{Table \ref{#1}}}

%Equations:
%1 equation:
\renewcommand{\eqref}[1]{\emph{Equation (\ref{#1})}}
%2 equations:
\newcommand{\eqrefTwo}[2]{\emph{Equation (\ref{#1})} and \emph{(\ref{#2})}}
%3 equations:
\newcommand{\eqrefThree}[3]{\emph{Equation (\ref{#1})}, \emph{(\ref{#2})} and \emph{(\ref{#3})}}
%4 equations:
\newcommand{\eqrefFour}[4]{\emph{Equation (\ref{#1})}, \emph{(\ref{#2})}, \emph{(\ref{#3})} and \emph{(\ref{#4})}}
%5 equations:
\newcommand{\eqrefFive}[5]{\emph{Equation (\ref{#1})}, \emph{(\ref{#2})}, \emph{(\ref{#3})}, \emph{(\ref{#4})} and \emph{(\ref{#5})}}
%6 equations:
\newcommand{\eqrefSix}[6]{\emph{Equation (\ref{#1})}, \emph{(\ref{#2})}, \emph{(\ref{#3})}, \emph{(\ref{#4})}, \emph{(\ref{#5})} and \emph{(\ref{#6})}}
%7 equations:
\newcommand{\eqrefSeven}[7]{\emph{Equation (\ref{#1})}, \emph{(\ref{#2})}, \emph{(\ref{#3})}, \emph{(\ref{#4})}, \emph{(\ref{#5})}, \emph{(\ref{#6})} and \emph{(\ref{#7})}}